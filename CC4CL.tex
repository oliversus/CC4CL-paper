\documentclass[amt]{copernicus}

\usepackage{hyperref}
\usepackage[capitalise,noabbrev]{cleveref}
\usepackage{natbib}
\usepackage{url}
\usepackage{lineno}
\usepackage{graphicx}
\usepackage{url}
\usepackage{array}
\usepackage{color}
\newcolumntype{P}[1]{>{\centering\arraybackslash}p{#1}}
\newcommand{\load}[1]{}

\begin{document}

% \chardef\_=`_
\linenumbers

\title{The Community Cloud retrieval for CLimate (CC4CL). Part I: A
  framework applied to multiple satellite imaging sensors.}

\author[1]{O. Sus}
\author[1]{M. Jerg}
\author[2]{C. Poulsen}
\author[2]{G. Thomas}
\author[1]{S. Stapelberg}
\author[3]{G. Mcgarragh}
\author[3]{A. Povey}
\author[1]{C. Schlundt}
\author[1]{M. Stengel}
\author[1]{R. Hollmann}

\affil[1]{DWD - Deutscher Wetterdienst, Frankfurter Straße 135, 63067 Offenbach, Germany}
\affil[2]{RAL Space - Rutherford Appleton Laboratory, Chilton, Didcot, OX11 0QX, UK}
\affil[3]{University of Oxford}

\runningtitle{CC4CL}

\runningauthor{O. Sus et al.}

\correspondence{O. Sus\\ oliver.sus@dwd.de}

\received{}
\pubdiscuss{} %% only important for two-stage journals
\revised{}
\accepted{}
\published{}

\firstpage{1}

\maketitle
% \sloppy

\begin{abstract}
  Hello abstract.
\end{abstract}

% ---------------------------------------------------------------------------
% outline
% 1) situation
%    - clouds in the climate system
%    - existing cloud property retrievals
%    - motivation of the Cloud CCI project
% 2) complication
%    - cloud remote sensing problems:
%      - overlap
%      - spatial extent vs. sensor resolution
%      - night-time vs day-time retrievals
%      - clouds over bright surfaces
%    - satellite data limitations:
%      - temporal coverage: AVHRR
%      - spatial resolution: MODIS, AATSR
%      - spectral resolution: MODIS, AATSR
%      - sounders (HIRS) and lidar (CALIOP on CALIPSO) vs imagers 
%      --> need to merge best of both worlds whilst accounting for
%          (inter-)calibration issues, orbital drift, changes in spectral resolution
%    - retrieval shortcomings
%      - availability of temporally consistent auxiliary datasets (e.g. BRDF,
%        atmospheric state, snow + sea-ice coverage)
%      - thresholding: simple, but no explicit radiative consistency between
%        observed radiances and retrieved cloud properties; not flexibly applicable
%        to new channels and/or sensors, no uncertainty quantification
%      - optimal estimation: complicated but sophisticated, accounting for
%        radiative consistency; however, local minimum problem; input background
%        uncertainties difficult to quantify; several OE-related assumptions
%        satisfied?
% 3) the CC4CL framework + key questions
%    - optimal estimation methodology applied as state-of-the-art approach within
%      cloud retrieval community
%    - the 4 advantages of CC4CL: consistency, simultaneity, uncertainty,
%      flexibility
%    - Are retrieval products as obtained from the various sensors consistent with
%      each other?
%    - How do CC4CL cloud property products compare to other existing retrieval
%      schemes and independent data sources?
% ---------------------------------------------------------------------------

\introduction

% 1) situation

%   clouds in the climate system 
Satellite data are an essential source of information in order to
better understand and predict climate change. They provide global long-term
observations from which geophysical parameters can be derived. These data are used for
time-series analysis of climate variables, and also for the assimilation into
or validation of climate models \citep{Comiso14,Yang13}. A paramount goal of
these efforts is the comprehensive characterization of the global energy and
water budgets.

Clouds modify atmospheric windows and radiative forcings of major greenhouse
gases \citep{Kiehl97}, and thus considerably constrain the global energy
budget. However, clouds are difficult to quantify in terms of composition and
temporal or spatial distribution. Observations of passive imagers do not
sufficiently resolve several important cloud properties, such as
vertical structure, sub-pixel heterogeneity, and the physical rather than radiatively
effective cloud boundary. Moreover, several background conditions (state of
surface and atmosphere, viewing geometry, sensor calibration and spectral
response uncertainties) complicate cloud retrievals. These
complications propagate uncertainties into derived cloud properties
themselves \citep{Hamann14}. Nonetheless, passive satellite imagers are the
most widely used instruments for cloud retrievals, providing global coverage at
acceptable cost. 

%   Existing cloud property retrievals
Examples for satellites based climatologies exploiting these types of sensors are the
International Cloud Climatology Project (ISCCP) \citep{Rossow99}, the
Pathfinder Atmosphere Extended (PATMOS-x) dataset
\citep{Heidinger09,Heidinger12}, and the EUMETSAT Satellite Application
Facility on Climate Monitoring (CM SAF) cLoud, Albedo and RAdiation (CLARA-A1)
dataset \citep{Karlsson13}.
Satellite observations of clouds are available for the past 40 years. However,
the production of climatologies and trend analyses is a complicated
task. Data need to be carefully processed and analysed in order
to derive a consistent long-term data record from several intercalibrated satellite
platforms. Consistency can be traded for continuity, and multi-platform
algorithms could exploit additional data when newer sensors become
available. Modern sensors provide improved spectral and spatial resolutions,
and thus potentially better cloud retrievals. However, their data records are too short to produce
climatologies of $>$ 30 years, and discontinuities are built into time series
when higher resolution satellite data are input to the processing.
Major complications of cloud retrievals are optically transparent clouds, multi-layer or overlapping clouds, and effective cloud top height determination. The degree to which these complications can be addressed depends on the nature of the retrieval and the type of input satellite data used. MODIS provides a much larger spectral resolution than just the six AVHRR heritage channels. MODIS and atmospheric sounders are clearly superior when detecting cloud height through the application of the ``CO2-slicing'' technique. However, when consistent climatologies are to be built, time series length and spatiotemporal resolution limit the choice in retrieval type and input satelllite data.  

\newpage

%   motivation of the Cloud CCI project
The European Space Agency has established the ESA Climate
Change Initiative program \citep{ESA_CCI_web,Hollmann13} in order to tackle
the problems outlined above and to advance the knowledge of the climate system.
The project's primary focus is the production of
thirteen Essential Climate Variables (ECVs) for the three domains ocean,
atmosphere, and land. ECVs are being produced for various climate drivers such as ozone, sea surface
temperature, ice sheets, and clouds. This study has been financed as part of the cloud ECV component of
ESA CCI \citep{ESA_Cloud_CCI_web}.
The main objective of ESA Cloud\textunderscore cci is to develop a state-of-the-art open source community
cloud retrieval algorithm, which is capable of processing passive imager data for a
number of \mbox{(non-)European} satellites covering several decades.
We used satellite data as retrieval input from MODIS Aqua and Terra (2000--2014) \citep{King92}, AVHRR on NOAA-7 to
NOAA-19 and METOPA (1978--2014) \citep{Jacobowitz03}, and AATSR on ENVISAT (?--?).
Only the AVHRR-equivalent channels from MODIS and AATSR are
used, thus the resulting retrieval data are henceforth referred to as the ``AVHRR heritage
dataset''. Moreover, the
resulting time series are carefully validated against well established existing climatologies (ISCCP, PATMOS-x, CM SAF, and MODIS
Collection 6), reanalysis and model data (ERA-Interim and EC-Earth),
ground-truth synoptic observations, and CALIPSO Lidar data.

The CC4CL core algorithm was developed in a modular fashion
and provides open source access to support distribution and further development within the
scientific community. Particular attention was paid to provide the flexibility of processing multiple instruments
with one and the same framework, thus maximising the consistency of
cloud products independent of the sensor source. Across the solar and thermal spectrum, the framework
accounts for physical and radiative consistency amongst all output variables and with
input satellite radiances. One key novel feature is the production of uncertainty
estimates of retrieval parameters through explict error propagation from input to output data. With these criteria in mind, the Oxford and Rutherford Aerosol and
Cloud (ORAC) retrieval \citep{Thomas09, Poulsen12} was chosen out of three competing algorithms within a ``Round Robin'' selection process. 

In this study, we present the key features of the CC4CL processing algorithm. We particularly focus on discussing the novel features of the framework, which set it apart from other approaches: the optimal estimation approach in general, the explicit uncertainty quantification through rigorous propagation of all known error sources to the final product, and the consistency of our long-term, multi-platform timeseries provided on various resolutions, from 0.5\textdegree\ up to 0.02\textdegree. Through describing all key input data and processing steps, we inform the future user about all relevant features of this new dataset, and its potential applicability in climate studies. We provide an overview of the retrieved and derived output variables. These are initially validated in a comprehensive and detailed analysis of retrieval results that we collocated with Calipso observations for three scenes in the Arctic and one scene in the Gulf of Guinea/West Africa. The results show that CC4CL provides very realistic estimates of cloud top height and cover for optically thick clouds, but produces mixed-layer estimates for cases where optically thin clouds overlap. % for a set of representative L2 output data scenes within a selected region against other retrieval algorithms and independent observations. 

% Key questions, hypotheses, analyses:

% Are the Cloud\textunderscore cci products derived from AVHRR NOAA18, MODIS AQUA, and AATSR comparable? Does CC4CL produce insignificant differences in retrieved variables despite of differences in spectral responses/LUTs? Analyse statistics of COT, CTP, cloud cover such as:
% \begin{itemize}
% \item histogram plot
% \item define distribution type
% \item mean, median, standard deviation, skewness, kurtosis
% \item analyse differences for statistical significane (T-test)
% \item if possible, analyse differences for bias and variance; plot residuals against CALIPSO variables to see if there are systematic differences
% \item residual analysis
% \item focus: we see more cloud fraction for MODIS in long term time series, is that confirmed in this scene? Possibly create new resolution data at 1km (MODIS orig.), 2km, 3km, 4km, 5km (AVHRR orig.) to see how cloud fraction changes. Refer to ``How small is a cloud'' paper.
% \end{itemize}

% Validation with CALIPSO: are there systematic biases? 
% \begin{itemize}
% \item analyse residuals between CALIPSO and Cloud\textunderscore cci variables, possibly also as a function of CTP or COT
% \end{itemize}


% PPS und CLARA
% Round Robin paper: tabellarische Auflistung der Unterscheide zwischen ORAC und anderen.
% MODIS C6?
% generell: welche retrieval gibt es, und was sind die Hauptunterschiede? Bayesian, decision trees, ...
% Stärken und Schwächen von Retrievalverfahren, Vergleichsstudien
% Unüberwindbare Schwächen: keine vertikale Struktur, Integral von LWP/IWP wird eigentlich nicht gesehen



\section{Data and methods}\label{input_data}

\subsection{L1 satellite data}\label{sec:L1_data}

\subsubsection{AVHRR}

The Advanced Very High Resolution Radiometer (AVHRR) is a cross-track scanner with a 2900 km swath width, providing almost daily global coverage. The sensor is equipped with six spectral channels (\autoref{tab:channels}), out of which only five can be transmitted simultaneously so that either channel 3a or 3b is available. In-flight calibration is performed only for thermal channels with a stable blackbody and a space view as references. AVHRR was/is mounted on several NOAA platforms as well as on EUMETSAT's MetopA/B, all of which are sun-synchronous, polar orbiting satellites. Due to a lack of orbit control technology for all NOAA AVHRR's, there is considerable orbit drift in equatorial crossing times (ECT) both for morning (ECT $<$ 12:00 LST) and afternoon (ECT $>$ 12:00 LST) satellites. To reduce drift induced changes in retrieved cloud properties, any AVHRR is replaced with its corresponding morning or afternoon successor once available ($=$ the AVHRR prime record). Typically, one morning and one afternoon NOAA satellite are in orbit simultaneously. 

For CC4CL, we use Global Area Coverage (GAC) L1c data on a reduced spatial resolution of 1.1 km $\times$ 4 km at nadir \citep{Devasthale16}. The AVHRR GAC L1c data record, including advanced intercalibration efforts, was produced for ESA Cloud\textunderscore cci and CMSAF \citep{Schulz09,Karlsson13}. CC4CL processed AVHRR data from 08/1981 (NOAA-7) up to 12/2014 (MetopA + NOAA-19). We applied a filtering technique to noisy channel 3b data (cite), and a database algorithm for splitting midnight orbits and blacklisting. 

\begin{table}[h]
  \caption{The CC4CL AVHRR-heritage dataset channel characteristics for AVHRR, AATSR, and MODIS.}
  \begin{tabular}{l|p{1.5cm}p{1.5cm}p{1.8cm}}
    \hline
    & CC4CL channel ID & sensor channel ID & channel width ($\mu m$) \\
    \hline
    AVHRR & 1 & 1 & 0.58 -- 0.68 \\
          & 2 & 2 & 0.725 -- 1.10 \\
          & 3 & 3a & 1.58 -- 1.64 \\
          & 4 & 3b & 3.55 -- 3.93 \\
          & 5 & 4 & 10.50 -- 11.50 \\
          & 6 & 5 & 11.5 -- 12.5 \\
    \hline
    MODIS & 1 & 1 & 0.62 -- 0.67 \\
          & 2 & 2 & 0.841 -- 0.876 \\
          & 3 & 6 & 1.628 -- 1.652 \\
          & 4 & 20 & 3.66 -- 3.84 \\
          & 5 & 31 & 10.78 -- 11.28 \\
          & 6 & 32 & 11.77 -- 12.27 \\
    \hline
    AATSR & 1 & 1 & 0.545 -- 0.565 \\
          & 2 & 2 & 0.649 -- 0.669 \\
          & 3 & 4 & 1.58 -- 1.64 \\
          & 4 & 5 & 3.51 -- 3.89 \\
          & 5 & 6 & 10.4 -- 11.3 \\
          & 6 & 7 & 11.5 -- 12.5 \\
    \hline
  \end{tabular}
  \label{tab:channels}
\end{table}


\subsubsection{MODIS}

The Moderate Resolution Imaging Spectroradiometer (MODIS) is carried by NASA's Terra and Aqua satellite platforms in a near sun-synchronous polar orbit at 705 km altitude. Due to orbit control, ECT is constant at 10:30 LST for Terra, and 13:30 LST for Aqua. The Aqua satellite is a member of the ``A-Train'' constellation, which also includes the CALIPSO and CloudSat satellites. MODIS is a cross-track scanner with a 2330 km swath width, producing a complete near-global coverage in less than two days \citep{Xiong09}.

CC4CL is applied to Collection 6 MOD021km (Terra) and MYD021km (Aqua) L1b input data \citep{MODIS_L1B}. For the AVHRR-heritage dataset produced here, the NASA Goddard space flight center performed the spectral subsetting of the 36 MODIS channels available (see \autoref{tab:channels} for the channels extracted), and data were directly shipped to ECMWF for archiving. The files are stored in HDF-EOS format at 1km spatial resolution, with the 250 m and 500 m channels having been aggregated to 1 km resolution. MODIS L1b data are organized in granules, each of which contains \texttildelow 5 minutes of MODIS data or \texttildelow 203 scan lines. Geolocation information is provided in separate files for Terra (MOD03) and Aqua (MYD03), containing geodetic latitude and longitude and solar/satellite zenith and azimuth angles. L1b data are corrected for all known instrumental effects through on-board calibrator data, and are organized into a viewing swath matching the geolocation file structure \citep{MODIS_PUG}. With CC4CL, we processed data from 02/2000 (Terra) or 08/2002 (Aqua) to 12/2014. 


\subsubsection{ATSR}

The second and third generation Along Track Scanning Radiometers (ATSR-2 and Advanced ATSR, \citet{Merchant12}) were launched on ESA's polar orbiting satellites ERS-2 and ENVISAT in 04/1995 and 03/2002, respectively. Both platforms were put into a sun-synchronous orbit at \texttildelow 780 km altitude, with ECT $=$ 10:30 LST for ERS-2 and ECT $=$ 10:00 for ENVISAT. Both ATSRs are identical in their overall configuration except for data transfer bandwidth (\autoref{tab:channels}). ATSR is designed to be self calibrating, with two on-board black-body targets for calibration of the thermal channels, and a sun-illuminated opal visible calibration target for the visible/near-infrared channels. ATSR uses a dual view system: a nadir view, and a forward view scanning the surface at an angle of 55\textdegree. The continuous scanning pattern produces a nadir resolution of approximately 1 km $\times$ 1 km with a swath width of 512 pixels or \texttildelow 500 km, providing global coverage every six days. 

We used no forward view data for cloud retrievals, as the 3-dimensional cloud structure produces parallax effects which are not accounted for within the current forward model. With CC4CL, we processed ATSR data from (operational?) launch date until 05/2003 (ERS-2) and 04/2012 (ENVISAT). 

\subsection{Auxiliary data}

\subsubsection{ERA-Interim}\label{sec:ERA-Interim}

We use ERA-Interim data as an apriori and first-guess input for the optimal estimation retrieval, and as input for the neural network cloud mask. ERA-Interim is a reanalysis of the global atmosphere, and is available from 1979 until today \citep{ERAInterim,Dee11}. The atmospheric profile variables are defined at 60 vertical levels. The original horizontal resolution is defined through a T255 spherical-harmonic representation for the basic dynamical fields, and through a reduced Gaussian grid with \texttildelow 79 km spacing fur surface fields. We downloaded ERA-Interim data from the ECMWF's MARS archive at a default spatial resolution of 0.72\textdegree, and at 0.1\textdegree\ for the neural network cloud mask input variables (\autoref{tab:ERA-Interim}). We acquired analysis (i.e. not forecast) data at 6-hourly timesteps. After download, all files were converted from GRIB to NetCDF format and remapped to the CC4CL preprocessor grid through CDO. This was necessary, as ERA-Interim coordinates are defined at the cell boundaries, whereas they are defined at the cell centres within CC4CL. The reanalysis data are temporally interpolated onto the satellite image's center time. Two ERA-Interim files before and after this time are linearly weighted as a function of their relative time differences.

ERA-Interim's land-surface model still needs to be improved in terms of its simulation of soil hydrology and snow cover. This affects the utilization of satellite data over land surfaces within ERA-Interim, which has negative effects on the representation of clouds and precipitation \citep{ERAInterim}. The confidence in temperature trend estimates however has improved considerably, so that ERA-Interim data have been used as an alternative to observational datasets to monitor climate change \citep{Willett10}.


\begin{table}[h]
  \caption{ERA-Interim variables used within CC4CL. Variables marked with * are available at 0.1\textdegree spatial resolution, all others default to 0.72\textdegree.}
  \begin{tabular}{p{3.8cm}|l|r|p{1.5cm}}
    \hline
    variable name & abbrev. & ID & unit \\
    \hline
    \multicolumn{4}{c}{profile variables} \\
    \hline
    Geopotential & Z & 129 & m$^2$ s$^{-2}$ \\
    Temperature  & T & 130 & K \\
    Specific humidity & Q & 133 & kg kg$^{-1}$ \\
    Log. surface pressure & LNSP & 152 & Pa \\
    Ozone mass mixing ratio & O3 & 203 & kg kg$^{-1}$ \\
    \hline
    \multicolumn{4}{c}{surface and single level variables} \\
    \hline
    Sea-ice cover* & CI & 31 & (0-1) \\
    Snow albedo & ASN & 32 & (0-1) \\
    Sea surface temperature & SSTK & 34 & K \\
    Total column water vapour & TCWV & 137 & kg m$^{-2}$ \\
    Snow depth* & SD & 141 & m of water equivalent \\
    10 metre U wind component & U10M & 165 & m s$^{-1}$ \\
    10 metre U wind component & V10M & 166 & m s$^{-1}$ \\
    2 metre temperature & T2M & 167 & K \\
    Land/sea mask & LSM & 172 & (0,1) \\
    Skin temperature* & SKT & 235 & K \\
    \hline
  \end{tabular}
  \label{tab:ERA-Interim}
\end{table}

\subsubsection{Land use}\label{sec:USGS}

We downloaded USGS Land Use/Land Cover raster data from the global land cover characteristics database \citep{USGS}. The USGS data are used as a land sea mask within the optimal estimation retrieval, as well as a land cover classificator within the cloud mask and the Pavolonis cloud typing scheme (update on this?). The dataset is defined on a regular lat/lon grid with 0.05\textdegree\ resolution. The USGS land cover classification was primarily derived from 1 km AVHRR NDVI 10-day composites for April 1992 through March 1993 \citep{USGS}. 

\subsubsection{Land surface BRDF}\label{sec:BRDF}

MODIS Collection 6 BRDF data (MCD43C1, \citet{MODIS_BRDF}), providing kernel weights for the RossThickLiSparseReciprocal BRDF model, are used within the retrieval scheme to set surface albedo and bidirectional reflectance distribution conditions. These data are available every 8 days derived from cloud-cleared 16-day Terra and Aqua measurements, and provided in HDF-EOS format at 0.05\textdegree\ spatial resolution. MCD43C1 data are classified as high quality given sufficient observations, and otherwise a low quality estimate is produced based on climatology anisotropy models. Validation against albedo measurements made at Baseline Surface Radiation Network (BSRN) sites show that the black-sky and white-sky albedo computed from the single sensor MCD43A1 high quality product are well within 5\% of the measured albedo, while the low quality product is within 10\% (copied from Greg, citation?).

We regridded MCD43C1 data to instrument resolution through bilinear interpolation, and filled missing pixels within the time series with pixel values of the temporally closest 8-day composite file providing valid data. For the pre-MODIS era, we produced a BRDF climatology by averaging all data available for a particular time slot. MCD43C1 kernel weights are applied to all CC4CL sensors, assuming negligible differences in spectral response functions. 

\subsubsection{Land surface emissivity}\label{sec:emissivity}

For land surface emissivity, we used the CIMSS global land surface IR emissivity database created by the Baseline Fit method \citep{Seemann08}. These data are derived from the MODIS operational land surface emissivity product (MOD11), to which the fit method is applied for filling spectral gaps between channels. CIMSS emissivity data are available on a monthly basis at ten wavelengths with 0.05\textdegree\ spatial resolution.

As for BRDF, we produced a land surface emissivity climatology for the pre-MODIS era by averaging all data available for a particular month.

\subsection{Calipso validation}\label{sec:calipso_method}

We resampled CC4CL L2 data to a regular latitude/longitude grid at 0.1\textdegree\ $\times$ 0.1\textdegree\ resolution (better 0.05?). This resampling is required for a intercomparison of CC4CL L2 data on a common grid, as differences in sensor spatial resolution are reduced when averaging all values available for each grid box. Calipso's Level 2 5 km Cloud Layer data were produced by averaging over $\approx$14 samples beams with 70 m diameter taken every 335 m within a 5 km along-track corridor. Thus, Calipso data have a 70 m across-track $\times$ 5 km along-track spatial resolution, and the size of corresponding CC4CL grid box is approximately 11 km meridionally $\times$ 2.9 to 5.6 km zonally. As a consequence, the CC4CL grid boxes are larger than the reference Calipso pixels, but are still small enough to resolve some of the cloud features that Calipso observes. Note that AVHRR GAC data were produced by averaging 5 neighbouring pixels across-track, but Calipso data were averaged along-track.

We found collocations between Calipso, AVHRR, MODIS, and AATSR for three study areas for 07/22/2008 19.15 LST (study area North America 1 = NA1), 07/22/2008 20:58 LST (NA2), and 07/27/2008 08.10 LST (Siberia = SIB). These are located within 60\textdegree\ to 75\textdegree\ N latitude, and contain vegetated land, snow-covered land, open ocean, and sea-ice cover surfaces (\cref{fig:RGB_07221915,fig:RGB_07222058,fig:RGB_07270810}). The Calipso track cuts through the three study areas almost orthogonally, so that about 130 collocated Calipso measurements are available per site. For NA1 and SIB, all Calipso pixels were classified as cloud covered, but for NA2, about half of the pixels are cloud free.




\section{The CC4CL retrieval system}\label{processing_chain}

\subsection{Heritage}

% Round robin (to 3)
In order to investigate different algorithms for their potential usage and
applicability in light of the aforementioned criteria, a rigorous and
comprehensive comparison effort was carried out at the onset of the
project.
Taking part in this investigation, termed ``Round Robin Exercise'',
were three different algorithms. The operational processing system of the
\citet{CMSAF_web}, the CLAVR-X algorithm used to generate the PATMOS-x
climatology \citep{Heidinger13}, and the Oxford and Rutherford Aerosol and
Cloud (ORAC) retrieval \citep{Thomas09, Poulsen12} which was previously used
to produce the GRAPE climatology \citep{Thomas09_GRAPE, GRAPE_web}.
Details of the assessment and the results can be found in
\citet{Stengel15}. All three algorithms were then driven with identical MODIS
and AVHRR input data and Era-Interim meteorological background information
from five days during the year 2008 and the results where analyzed with
respect to CLOUDSAT, Calipso and AMSR-E reference data.
This study not only led to a number of recommendations for further retrieval
development but also unveiled that there was no clear winner outperforming the
other algorithms.

\subsection{Preprocessing}

\begin{figure}[h]
  % \centering
  \includegraphics[width=0.47\textwidth]{figures/outline_preprocessor.png}
  \caption{Schematic of the CC4CL preprocessor.}
  \label{fig:CC4CL_preprocessor}
\end{figure}

The CC4CL preprocessor initially defines the dimensions and content of the sensor, surface, and preprocessing grids (\autoref{fig:CC4CL_preprocessor}).

The sensor and surface grids have the same extent and resolution as the input orbit or granule. The sensor grid is filled with sensor radiances and angles (section \ref{sec:L1_data}), whereas data on surface BRDF (section \ref{sec:BRDF}), snow/ice coverage (from ERA-Interim, section \ref{sec:ERA-Interim}), and surface emissivity (section \ref{sec:emissivity}) are bilinearly interpolated onto the surface grid. We use BRDF data over land only. For sea pixels, the Cox/Munk ocean surface reflectance model calculates BRDF coefficients as a function of ERA-Interim wind speed. The albedo of snow/ice covered pixels is set to globally constant values of 0.958 (Ch1), 0.868 (Ch2), 0.0364 (Ch3), and 0.0 (Ch4), and is area weighted in case of fractional sea/ice cover.

The preprocessing grid is a regular latitude/longitude grid that covers the extent of the sensor grid, but is defined at a coarser resolution of 0.72\textdegree $\times$ 0.72\textdegree. The preprocessor calculates geolocation values for each grid box, and fills it with data: the average of all sensor angle and surface emissivity values falling into a grid box, and spatially interpolated (nearest neighbour) USGS data (section \ref{sec:USGS}). ERA-Interim variables are defined as surface and profile variables, and were transformed before input to the preprocessing grid as described in section \ref{sec:ERA-Interim}. For profile variables, vertical geopotential coordinates are calculated from pressure coordinates.

The preprocessor then calls the cloud mask (section \ref{sec:CloudMask}) and the cloud typing (section \ref{sec:Pavolonis}) algorithms, and writes output data to the surface grid. Finally, RTTOV is executed on the preprocessing grid data as defined by ERA-Interim surface and profile variables. RTTOV outputs above and below cloud transmission for shortwave channels, and radiances (above cloud up- and downwelling, below cloud upwelling), transmission (above and below cloud), and emissivity for longwave channels.

All data are written to NetCDF files as subsequent input to the main processor ORAC. As we determine the cloud mask and phase in the preprocessor, ORAC can optionally be run only on cloudy pixels and one phase to reduce processing time.

\subsection{CC4CL cloud retrieval}

\subsubsection{Cloud masking}\label{sec:CloudMask}

CC4CL applies a set of artificial neural networks (ANN) for cloud masking, one for each of the illumination conditions day, night, and twilight. The ANNs are multilayer perceptrons with one input layer, one hidden layer with 50 neurons, and one output layer, which produces pseudo CALIOP cloud optical depth (ANNCOD) ranging from 0 to 1.

\begin{table}[h]
  \caption{Threshold values applied to ANNCOD data for cloud mask classification.}
  \begin{tabular}{llllll|l} %{P{0.3cm}P{0.3cm}P{0.7cm}P{0.3cm}P{0.2cm}P{1cm}|P{0.7cm}}
    \hline
    day & night & twilight & land & sea & snow/ice & threshold \\
    \hline
    % day
    x   &       &          & x    &     &          & 0.2  \\
    x   &       &          & x    &     & x        & 0.35 \\
    x   &       &          &      & x   &          & 0.1  \\
    x   &       &          &      & x   & x        & 0.4  \\ \hline
    % night
    & x     &          & x    &     &          & 0.3  \\
    & x     &          & x    &     & x        & 0.35 \\
    & x     &          &      & x   &          & 0.2  \\
    & x     &          &      & x   & x        & 0.4  \\ \hline
    % twilight
    &       & x        & x    &     &          & 0.3  \\
    &       & x        & x    &     & x        & 0.4  \\
    &       & x        &      & x   &          & 0.35 \\
    &       & x        &      & x   & x        & 0.4  \\ \hline
  \end{tabular}
  \label{tab:ANN_thresholds}
\end{table}

The various ANNs were trained with NOAA-18 AVHRR L1c data, auxiliary information, and cloud optical depth (COD) ``truth'' data obtained from the CALIPSO satellite's LIDAR called CALIOP at 532 nm (\mbox{CAL\textunderscore LID\textunderscore L2\textunderscore 05kmCLay-Prov-V3-01}). AVHRR Ch3a data were generally excluded. We trained the day ANN with all remaining AVHRR channels, but also excluded Ch3b to be consistent with those NOAA platforms that switch between Ch3b transmission at night and Ch3a at day (NOAA-16, NOAA-17, MetopA). For night and twilight conditions, we produced ANNs both with and without Ch3b data input. This was necessary to avoid misclassification of very cold clouds and/or land surfaces due to Ch3b's very low signal-to-noise ratio. Next to the round robin days (?), we selected 12 additional representative training days in 2008 that satisfy basic requirements in terms of collocation between NOAA-18 and CALIPSO, representation of cloud optical depth seasonality, and global coverage. Prior to training, CALIPSO COD values were set to be $\le$ 1. Auxiliary data input are the ERA-Interim skin temperature, a snow/ice mask derived from ERA-Interim snow depth and sea ice concentration, and the USGS land/sea mask. Finally, we applied a simple correction algorithm to remove a viewing-angle dependency of retrieved ANNCOD.

The binary cloud mask is finally estimated by classification of ANNCOD data into clear and cloudy through a set of threshold values. The thresholds themselves vary depending on illumination and surface conditions, namely land, sea, and snow/ice cover (\autoref{tab:ANN_thresholds}). As the ANN was trained with AVHRR data only, differences in spectral response functions need to be considered before the ANN can be applied to MODIS and AATSR. We derived appropriate coefficients through linear regression analysis between collocated satellite observations for each input channel pair (\autoref{tab:ANN_coefficients}). The resulting coefficients were applied to MODIS and AATSR satellite data before ANN input.

\begin{table}[h]
  \caption{Linear regression coefficients between collocated AVHRR and MODIS/AATSR channels.}
  \begin{tabular}{l|l|l} %{P{0.3cm}P{0.3cm}P{0.7cm}P{0.3cm}P{0.2cm}P{1cm}|P{0.7cm}}
    \hline
    CC4CL channel ID & sensor & regression coefficients \\
    \hline
    1 & MODIS & 0.8945 $\times$ ch1 + 2.217 \\
    & AATSR & 0.8542 $\times$ ch1 \\ \hline
    2 & MODIS & 0.8336 $\times$ ch2 + 1.749 \\
    & AATSR & 0.7787 $\times$ ch2 \\ \hline
    4 & MODIS & 0.9944 $\times$ ch4 + 1.152 \\
    & AATSR & 1.0626 $\times$ ch4 - 15.777 \\ \hline
    5 & MODIS & 0.9742 $\times$ ch5 + 7.205 \\
    & AATSR & 0.9793 $\times$ ch5 + 5.366 \\ \hline
    6 & MODIS & 0.9676 $\times$ ch6 + 8.408 \\
    & AATSR & 0.9838 $\times$ ch6 + 4.255 \\
    \hline
  \end{tabular}
  \label{tab:ANN_coefficients}
\end{table}

We estimate cloud mask uncertainty based on the assumption that this uncertainty is inversely proportional to the difference between retrieved ANNCOD and the threshold applied. As a first step, we quantified a ``truth'' uncertainty or, rather, error. To do so, we generated a CALIOP cloud mask by application of a clear/cloudy threshold value of 0.05. The CALIOP cloud mask is then compared with the collocated ANN mask by quantification of a Percent Correct (PEC) score. PEC is basically the ratio between all correctly classified pixels and the number of all pixels analysed. Finally, the ``truth'' uncertainty is defined as 100 $-$ PEC \%. We then established the statistical relationship between this uncertainty and the ANNCOD difference to its threshold. Before application of the approach, we normalised differences (ND) to 1. We found a linear correlation between uncertainty and ND for clear cases, and a second order polynomial correlation for cloudy cases (\autoref{fig:NN_unc}):

clear:
\begin{equation}
  y = 37.275 \times ND + 49.2
\end{equation}

cloudy:
\begin{equation}
  y = 54.133 \times (ND-1)^2 + 1.862
\end{equation}

The equations of these regression fits are used within CC4CL to quantify cloud mask uncertainty as a function of ND.

\begin{figure}[h]
  % \centering
  \includegraphics[width=0.47\textwidth]{figures/NN_uncertainty_EQ.png}
  \caption{Neural network cloud mask uncertainty}
  \label{fig:NN_unc}
\end{figure}

\subsubsection{Cloud typing}\label{sec:Pavolonis}

Cloud phase is determined by application of the Pavolonis cloud typing algorithm \citep{Pavolonis05}. The Pavolonis algorithm outputs 6 cloud types (\cref{tab:cloud_types}), which we then reclassified into water or ice clouds: liquid = fog/warm liquid/supercooled, ice = opaque ice/cirrus/overlap. For CC4CL, the fog type test was deactivated. The algorithm always uses the 0.65-, 11-, and 12-$\mu m$ channel data. It is further designed to read 3.75$\mu m$ data whenever available, and otherwise switches to a 1.65$\mu m$ mode. These two different approaches produce nearly identical results, exept for certain thin clouds and cloud edges \citep{Pavolonis05}. In addition, we introduced two new cloud types within CC4CL. We decided to change phase for ice clouds whose retrieval CTT is to high (new cloud type = SWITCHED\textunderscore TO\textunderscore WATER if CTT $>$ 273.16 K, the freezing point of water), and for water clouds whose CTT is too low (SWITCHED\textunderscore TO\textunderscore ICE if CTT $<$ 233.16 K, the lower limit of supercooled water).

The Pavolonis algorithm has weaknesses in detecting cirrus clouds at high latitudes, which are often misclassified as opaque ice clouds. Performance is considerably better when the also available VIIRS algorithm is used, which provides additional channels and threshold tests. However, these cannot be applied to our AVHRR heritage dataset \citep{Pavolonis05}.

\begin{table}[h]
  \caption{Cloud type classification for CC4CL and Calipso.}
  \begin{tabular}{l|l||l|l}
    \hline
    ID & CC4CL & ID & Calipso \\
    \hline
    0 & clear & 0 & low transparent \\
    1 & switched to water & 1 & low opaque \\
    2 & fog & 2 & stratocumulus \\
    3 & water & 3 & low broken cumulus \\
    4 & supercooled & 4 & altocumulus \\
    5 & switched to ice & 5 & altostratus \\
    6 & opaque ice & 6 & cirrus \\
    7 & cirrus & 7& deep convective \\
    8 & overlap & 8 & n/a \\
    % 9 & probably opaque ice & n/a \\
    % 10 & probably clear & n/a \\
    \hline
  \end{tabular}
  \label{tab:cloud_types}
\end{table}
\subsubsection{Optimal estimation retrieval of COT, CER and CTP}
The optimal estimation retrieval, the Oxford RAL Aerosol and Cloud (ORAC) algorithm, is conceptually based on an inverse problem. The forward model simulates top of atmosphere satellite radiance as a function of atmospheric state and a cloud model. Then, an inverse model attempts to find the optimal fit between the simulated and observed TOA radiances by incrementally varying the parameters of the cloud model. Simulations and measurements are weighted according to measurement errors and uncertainties regarding prior knowledge and the forward model. The important benefits of ORAC are that cloud parameters are retrieved using information in all satellite channels simultaneously, so that the retrieved parameters provide a robust representation of the short-wave and long-wave radiance effects of the observed cloud. Moreover, the algorithm quantifies retrieval uncertainty, which is as a measure of the consistency between the retrieved cloud parameters and the satellite measurments \citep{Poulsen12}. For a more detailed description of the ORAC algorithm see part II of this publication (add citation).

\subsection{Post-processing}
For each input pixel, the main processor produces retrieval values for both ice and liquid clouds. The postprocessor will then select the appropriate output variables according to the Pavolonis cloud phase. As described in section \ref{sec:Pavolonis}, the postprocessor changes cloud phase in case retrieved CTT does not match the Pavolonis phase. Finally, output variables are written to primary and secondary NetCDF files.

\begin{table*}[h]
  \caption{CC4CL primary and secondary output. NN = neural network, CV = control vector, PP = postprocessed, PV = \citet{Pavolonis05} algorithm.}
  \begin{tabular}{l|l|l|l|l}
    \hline
    variable name & abbrev. & unit & origin & description \\
    \hline
    \multicolumn{5}{c}{primary variables} \\
    \hline
    cloud mask & cldmask & 1 & NN & Binary cloud occurrence classification \\
    cloud type & cldtype & 1 & PV & Categorical cloud type classification \\
    cloud phase & phflag & 1 & PV &  cloud phase classification \\
    cloud top pressure & ctp & hPa & CV & OE retrieval estimate of cloud top pressure \\
    cloud top pressure unc. & ctp\_unc & hPa & CV & OE retrieval unc. of cloud top pressure \\
    cloud effective radius & cer & mic & CV & OE retrieval estimate of cloud effective radius \\
    cloud effective radius unc. & cer\_unc & mic & CV & OE retrieval unc. of cloud effective radius\\
    cloud optical thickness & cot & 1 & CV & OE retrieval estimate of cloud optical thickness \\
    cloud optical thickness unc. & cot\_unc & 1 & CV & OE retrieval unc. of cloud optical thickness \\
    surface temperature & stemp & kelvin & CV & OE retrieval estimate of surface temperature \\
    surface temperature unc. & stemp\_unc & kelvin & CV & OE retrieval unc. of surface temperature\\
    \hline
    \multicolumn{5}{c}{secondary variables} \\
    \hline
    cloud mask unc. & cldmask\_unc & 1 & PP & derived from NN output and threshold distance \\
    cloud top height & cth & km & PP & derived from CTP and atmospheric profile \\
    cloud top height unc. & cth\_unc & km & PP & derived from retrieval unc. of CTP \\
    cloud top temperature & ctt & kelvin & PP & derived from CTP and atmospheric profile \\
    cloud top temperature unc. & ctt\_unc & kelvin & PP & derived from retrieval unc. of CTP \\
    cloud water path & cwp & g/m$^2$ & PP & derived from CER and COT \citep{Nakajima90} \\
    cloud water path unc. & cwp\_unc & g/m$^2$ & PP & derived from retrieval unc. of CER and COT \\
    cloud albedo at 0.06 µm & cla & 1 & PP & derived from CER and COT based on ? \\
    cloud albedo at 0.06 µm unc. & cla\_unc & 1 & PP & derived from retrieval unc. of CER and COT \\
    cloud albedo at 0.08 µm & cla & 1 & PP & derived from CER and COT based on ? \\
    cloud albedo at 0.08 µm unc. & cla\_unc & 1 & PP & derived from retrieval unc. of CER and COT \\
    cloud effective emissivity & cee & 1 & PP & derived from 10.8 and 12.0 mic ? \\
    \hline
  \end{tabular}
  \label{tab:L2_variables}
\end{table*}


\section{L2 data - analysis and initial validation}\label{L2_data}

The analysis and initial validation of L2 data is twofold. We first examine CC4CL cloud properties for one sample scene that extends approximately from 100\textdegree\ W to 170\textdegree\ W and 45\textdegree\ N to 75\textdegree\ N over North America. We focus on the consistency of retrieval values derived from different sensors (AVHRR, MODIS, AATSR). This also includes pixel-based uncertainties of the key variables ctp, cot, cer, and cldmask. We then perform an initial validation of retrieved cloud properties, for which CALIOP-CALIPSO data are our truth reference. This validation is limited to three high-latitude scenes for which collocations for all sensors with CALIOP are available.

% In this section we firstly show and discuss the cloud properties generated by CC4CL for one example scene. A particular focus here is the consistency among the sensors AVHRR, MODIS and AATSR. For the same scene we also present and discuss the pixel-based uncertainties of the core variables CMA, CTP, COT and CER. Secondly, a qualitative assessment of the retrieved cloud properties is carried out for three scenes using CALIOP-CALIPSO as reference.
% It needs to be noted that for matching the thee sensors mentioned above and CALIOP we are limited to the high-latitudes.
 
\subsection{CC4CL cloud properties}

The sample scene is characterized by various cloud types, and the CC4CL cloud mask defines only a relatively small fraction as cloud free (\cref{fig:CTP_intercomparison,fig:COT_intercomparison,fig:CER_intercomparison}).  When visually inspecting CC4CL output for CTP, COT, and CER, the retrieval data appear highly similar for the three different sensors. Main spatial patterns are resolved in all products. The data show that there are more cloud free AVHRR pixels, which is related to the coarser spatial resolution compared to MODIS and AATSR. The LST difference ranges from XY to XY minutes, and thus there is little cloud displacement between observation times.
 
CTP data are approximately normally distributed for all three sensors. Both COT and CER show positive kurtosis and skewness, as values close to 0 are most frequent. CER data are somewhat bimodal, having a primary peak at $\approx$12 micron and a secondary peak at $\approx$35 micron (\cref{fig:histograms,tab:retrieval_statistics}). Mean value differences are not significant between AVHRR and MODIS for CTP, MODIS and AATSR for COT, and AVHRR and AATSR for CER. The standard deviation of differences between two sensors are always lowest for AVHRR minus MODIS (\cref{tab:retrieval_statistics}). Significance tests of mean differences and standard deviations of residuals between sensor retrievals are sensitive to outliers. These are to some extent influenced by cloud displacement due to observation time differences. Even though we found no significant relationship between sensor retrieval residuals and observation time difference (data not shown), residuals are likely to be smaller and thus possibly insignificant if sensor obervation times were identical.
 
\subsection{Uncertainties}

Median absolute uncertainties are CTP = \load{NA2_ctp_unc_median}26.7 hPa, COT = \load{NA2_cot_unc_median}6.1, CER = \load{NA2_cer_unc_median}2.0 mic, and cloud mask = \load{NA2_cmask_unc_median}13.7 \% (\cref{fig:uncertainties}). The median relative retrieval uncertainty (not shown) is relatively low for all three retrieval variables (CTP = \load{NA2_ctp_unc_median_relative}4.7 \%, COT = \load{NA2_cot_unc_median_relative}6.1 \%, CER = \load{NA2_cer_unc_median_relative}2.0 \%). COT uncertainties increase with COT magnitude, and the RGB image (\cref{fig:RGB_07222058}) shows that largest uncertainties are found in cases of opaque cloud coverage and probably cloud over sea-ice surfaces. CER results are similar to COT, although relative uncertainties are somewhat lower. Cloud free areas show increased cloud mask uncertainties, and in particular over sea-ice surface areas. Note that the cloud mask uncertainties have been quantified as a function of the normalized difference to the cloud mask threshold, whereas relative retrieval uncertainties (100 $\times$ uncertainty $\div$ retrieved value) are shown for CTP, COT, and CER. 

\begin{table*}[t]
  \caption{Statistics of CTP, COT, and CER retrieval values for study area NA2 and AVHRR (first value in each cell), MODIS (second value), and AATSR (third value). $\Delta$ values are given for AVHRR minus MODIS (first value in each cell), AVHRR minus AATSR (second value), and MODIS minus AATSR (third value). $^{\ast}$t-Test p-value $>$ 0.1, indicating that differences in mean values are not significant.}
  \begin{tabular}{l|lllll}
    \hline
                 & mean & median & stddev & skewness & kurtosis \\
    \hline
             CTP & \load{ctpMeanN18}667.2, \load{ctpMeanMYD}665.0, \load{ctpMeanENV}645.2 & \load{ctpMedN18}667.8, \load{ctpMedMYD}668.1, \load{ctpMedENV}632.4 & \load{ctpStdN18}147.5, \load{ctpStdMYD}142.7, \load{ctpStdENV}146.2 & \load{ctpSkewN18}-0.2, \load{ctpSkewMYD}-0.2, \load{ctpSkewENV}0.1 & \load{ctpKurtN18}-0.4, \load{ctpKurtMYD}-0.4, \load{ctpKurtENV}-0.8 \\
    $\Delta$ CTP & \load{ctpdMeanN18}2.2$^{\ast}$, \load{ctpdMeanMYD}21.9, \load{ctpdMeanENV}19.7 & \load{ctpdMedN18}4.2, \load{ctpdMedMYD}22.3, \load{ctpdMedENV}18.5 & \load{ctpdStdN18}63.0, \load{ctpdStdMYD}138.7, \load{ctpdStdENV}138.9 & \load{ctpdSkewN18}-0.4, \load{ctpdSkewMYD}-0.3, \load{ctpdSkewENV}-0.3 & \load{ctpdKurtN18}8.2, \load{ctpdKurtMYD}1.0, \load{ctpdKurtENV}0.7 \\
             COT & \load{cotMeanN18}12.3, \load{cotMeanMYD}13.6, \load{cotMeanENV}13.4 & \load{cotMedN18}7.2, \load{cotMedMYD}8.6, \load{cotMedENV}8.8 & \load{cotStdN18}19.8, \load{cotStdMYD}19.7, \load{cotStdENV}17.6 & \load{cotSkewN18}6.6, \load{cotSkewMYD}5.7, \load{cotSkewENV}5.3 & \load{cotKurtN18}60.5, \load{cotKurtMYD}46.2, \load{cotKurtENV}40.8 \\
    $\Delta$ COT & \load{cotdMeanN18}-1.3, \load{cotdMeanMYD}-1.2, \load{cotdMeanENV}0.2$^{\ast}$ & \load{cotdMedN18}-0.6, \load{cotdMedMYD}-1.2, \load{cotdMedENV}-0.5 & \load{cotdStdN18}16.5, \load{cotdStdMYD}22.0, \load{cotdStdENV}21.3 & \load{cotdSkewN18}0.7, \load{cotdSkewMYD}2.4, \load{cotdSkewENV}1.8 & \load{cotdKurtN18}59.6, \load{cotdKurtMYD}41.5, \load{cotdKurtENV}33.1 \\
             CER & \load{cerMeanN18}21.1, \load{cerMeanMYD}19.2, \load{cerMeanENV}21.3 & \load{cerMedN18}16.5, \load{cerMedMYD}14.4, \load{cerMedENV}18.1 & \load{cerStdN18}13.0, \load{cerStdMYD}12.1, \load{cerStdENV}10.9 & \load{cerSkewN18}1.1, \load{cerSkewMYD}1.4, \load{cerSkewENV}0.6 & \load{cerKurtN18}1.4, \load{cerKurtMYD}1.2, \load{cerKurtENV}-0.8 \\
    $\Delta$ CER & \load{cerdMeanN18}1.9, \load{cerdMeanMYD}-0.2$^{\ast}$, \load{cerdMeanENV}-2.1 & \load{cerdMedN18}0.5, \load{cerdMedMYD}-1.0, \load{cerdMedENV}-1.9 & \load{cerdStdN18}7.0, \load{cerdStdMYD}11.6, \load{cerdStdENV}11.3 & \load{cerdSkewN18}0.8, \load{cerdSkewMYD}0.8, \load{cerdSkewENV}0.5 & \load{cerdKurtN18}7.9, \load{cerdKurtMYD}4.4, \load{cerdKurtENV}2.3 \\
    \hline
  \end{tabular}
  \label{tab:retrieval_statistics}
\end{table*}


\begin{figure*}[h]
  %\centering
  \includegraphics[width=\textwidth]{figures/07222058_ctp_multi.png}
  \caption{CTP retrieval values for study area NA2 with data from AVHRR (left), MODIS (middle), and AATSR (right).}
  \label{fig:CTP_intercomparison}
%\end{figure*}

%\begin{figure*}[h]
  %\centering
  \includegraphics[width=\textwidth]{figures/07222058_cot_multi.png}
  \caption{COT retrieval values for study area NA2 with data from AVHRR (left), MODIS (middle), and AATSR (right).}
  \label{fig:COT_intercomparison}
%\end{figure*}

%\begin{figure*}[h]
  %\centering
  \includegraphics[width=\textwidth]{figures/07222058_cer_multi.png}
  \caption{CER retrieval values for study area NA2 with data from AVHRR (left), MODIS (middle), and AATSR (right).}
  \label{fig:CER_intercomparison}
\end{figure*}

\begin{figure*}[h]
  %\centering
  \includegraphics[width=0.68\textwidth]{figures/07222058_uncertainties_absolute.png} %{figures/07222058_uncertainties_percent_goodspacing.png}
  \caption{Absolute uncertainties of MODIS AQUA retrieval data for study area NA2 and CTP [hPa], COT, CER [mic], and Cloud mask [\%].} %{Relative uncertainties [\%] of MODIS AQUA retrieval data for study area NA2 and CTP, COT, CER, and Cloud mask.}
  \label{fig:uncertainties}
\end{figure*}

\begin{figure*}[h]
  %\centering
  \includegraphics[width=\textwidth]{figures/07222058_histograms.png}
  \caption{Density histograms of NOAA18, MODIS AQUA, and AATSR retrieval data for study area NA2 and (a) CTP, (b) CTP differences, (c) COT, (d) COT differences, (e) CER, and (f) CER differences.}
  \label{fig:histograms}
\end{figure*}

\subsection{Validation with Calipso}

We found collocations between Calipso, AVHRR, MODIS, and AATSR for three study areas in the Arctic at 07/22/2008 19.15 LST (study area North America 1 = NA1, n = \load{NA1_length}120, \cref{fig:RGB_07221915}), 07/22/2008 20:58 LST (NA2, n = \load{NA2_length}163, \cref{fig:RGB_07222058}), and 07/27/2008 08.10 LST (Siberia = SIB, n = \load{SIB_length}116, \cref{fig:RGB_07270810}). These are located within 60\textdegree\ to 75\textdegree\ N latitude, and contain vegetated land, snow-covered land, open ocean, and sea-ice surfaces. For NA1 and SIB, all Calipso pixels were classified as cloud covered, but for NA2, about half of the pixels are cloud free.

When including AATSR, collocations are restricted to high latitude areas and the few pixels located at the intersection between Calipso and AATSR. We thus decided to extend the validation analysis by another scene without AATSR data, which we found in the Gulf of Guinea/West Africa between 7\textdegree\ S and 12\textdegree\ N at 24/10/2009 13.45 LST (Africa = AFR, n = \load{AFR_length}1181, \cref{fig:RGB_10241345}). For this scene, about ten times more pixel values are available than for the other scenes, displaying cloud systems such as low-level stratocumulus and deep convection that are not contained in the Arctic data shown here.

\subsubsection{Case studies}

We divided all study areas into logical sectors, for each of which a characteristic pattern of cloud coverage and type predominates. 

\vspace{5mm}\underline{\textit{Case study NA1}}\vspace{2mm}

Study area NA1 is a completely cloud covered scene over northern Canada containing land, ice-covered land, and open ocean surfaces. There is a mix of single and multi-layered clouds of variable optical thickness and height. 
CC4CL correctly classifies pixels as cloud covered, except for a few cases in sectors 3 and 4. CTH retrievals are very consistent among the three sensors, and only differ in sector 2. Clouds are generally retrieved too low compared to Calipso's top layer, unless the latter is optically thick as in sector 4. In the case of a (semi-)transparent cloud top layer, multiple surfaces contribute to the observed satellite data. CC4CL CTH is then located closer to (sector 1), at (sector 3), or even below the underlying cloud layer (sector 2). For single layer, optically thick (COT $>$ 1) cloud coverage however, CC4CL and Calipso CTH agree very well (sector 4). Under such conditions, the retrieval of CTP and derived products using AVHRR heritage channel data is very accurate. Cloud phase agreement between CC4CL and Calipso is very variable. It is best for optically thick high ice cloud coverage (sector 1), and worst for low water clouds (sector 4).

% For this scene we defined 4 sectors. In sector 1, a continuous, high cloud field with CTH around 9 km is found in CALIOP. This cloud layer consists of ice and is optically thin, i.e. the COT does not exceed 1, with a tendency of getting optically thinner towards the end of the sector. For some pixels a second cloud layer is found with CTH around 4 km, also mainly being ice. In addition some thin clouds are scattered a bit higher than the second cloud layer or (towards the end of the sector) in the lower troposphere. The latter being water clouds. In sector 2, the uppermost clouds are geometrically even thinner, with COT also not exceeding 1. In the last third of this sector, some optically thicker low-level clouds are present. Sector 3, similar to sector 1 shows an optically thin high cloud layer and a second layer 2 to 3 km below in which a total optical depth of 1 is exceeded. At the end of sector 3 the uppermost layer gets optically even thinner, while the second cloud layer disappears and a new second layer in the lower troposphere is present. Sector 4 is mainly characterized by single cloud layers, of optical thickness larger than 1, at 2km and 4km, respectively. 

% The CC4CL cloud detection identifies nearly all cloudy pixels for all three sensors. Only the end of sector 3 (all three sensors) and in sector 4 (only AVHRR) some few cloudy pixels are missed. Except in sector 2, the CC4CL CTP retrievals are mostly very consistent among the three sensors. However, the clouds are generally retrieved too low compared to the uppermost cloud layer top in CALIOP, if the uppermost layer is optically thin. In these conditions, the cloud top is usually placed between the optically thin top layer and the underlying layer with a tendency towards the lower layer. If no second cloud layer is present, the CTP is put close to the surface (e.g. in sector 2). In sector 3 the uppermost cloud layer is very thin, often below 0.15 COT), while together the second layer the COT reaches COT>1, leading to placing the CTP nearly directly at the vertical location of the second layer for all sensors.
% In contrast to sectors 1-3, in sector 4 the CC4CL CTP is in very good agreement with the location of the top of the uppermost CALIOP cloud layer. These layer are optically thicker (COT>1) and are not overlaid by thinner layers, supporting a very accurate CTP retrieval with CC4CL applied to the passive sensors.
% The agreement of the CC4CL phase with CALIOP is good in sector 1 until the presents of a very thin, high single layer cloud, where CC4CL switches to water phase while CALIOP has ice. The phase in sector 2 is very variable but some agreement of CC4CL to CALIOP can be found. In sector 3, the uppermost cloud layer is ice in CALIOP, to which the CC4CL phases mostly agree. Towards the second layer CALIOP gives liquid, while CC4CL mainly remains at the ice phase, however, with some liquid pixels inclusions. In sector 4 all clouds are liquid in CALIOP. CC4CL gives ice clouds for the second part of that sector, when the cloud top is near 4km. CTT retrievals show….K?! Generally speaking, the retrieved cloud phases among the passive sensors agree well.

\vspace{5mm}\underline{\textit{Case study NA2}}\vspace{2mm}

Study area NA2 is located entirely over snow/ice free land in Western Canada. Calipso cloud coverage is \load{cfree_calipso}47.2\%, spatially broken, and variable in height and phase. Clear sky pixels are mostly identified by CC4CL (\load{cfree_cc4cl_and_calipso_NA2}68.8\% correct), and cloudy pixels are rarely missed (\load{cloudy_cc4cl_and_calipso_NA2}78.7\% correct). CC4CL retrievals of thin high clouds and false positive cloudy pixels have low CTH values (sector 1). Calipso's small-scale horizontal variability in cloud phase is reflected by CC4CL data, which however overestimate the fraction of liquid water clouds in sector 2. CC4CL reproduces Calipso's spatial variability in CTH, which it only slightly underestimates by 0.5--1 km in sector 2. In sector 3 however, CC4CL considerably underestimates CTH by up to 7 km. Most of these clouds are optically and geometrically thin.

% Case 2 is located again over North America. First half (sector 1) of this case is mainly cloud free except some high, optically very thin clouds at 9,5km in the beginning. Sectors 2 and 3 contain mainly broken cloud fields that actually do show a lot of along-track variability. Clouds are vertically located around 4km in sector 2 and higher (mainly between 6 and 8km) in sector 3.

% Except from some scattered clouds in AATSR and MODIS, the CC4CL sensors generally reflect the clear-sky situations in sector 1. The high, very thin cloud layer at the beginning is missed. The obviously wrongly identified clouds in this sector are mostly low-level according to the CTP retrieval. In sector 2 all clouds are correctly identified, however, some small gaps seen in the CALIOP data are not identified by the CC4CL at all sensors. Only the AVHRR product does reflect them to some extend. The horizontal inhomogeneity in also found for all three passive sensors in terms of small-scale variabilities of cloud phase. 

% Generally speaking, CC4CL gives too often liquid phase in sector 2. The CC4CL CTP retrievals are very variable, as in CALIOP, with an underestimation of the cloud height of approx. 05. to 1km on average. The horizontal inhomogeneity in the cloud fields in sector 3 is similar to sector 2. The CC4CL results are similarly inhomogeneous. Even though most clouds are correctly identified by CC4CL in sector 3, the retrievals of phase and cloud top pressure are significantly deviating from CALIOP. The cloud top pressure difference between CALIOP and CC4CL partly amount to 7km.

\vspace{5mm}\underline{\textit{Case study SIB}}\vspace{2mm}

Study area SIB crosses the Novaya Zemlya islands north of Siberia and is defined by a mixture of open ocean and partially snow/ice covered land surfaces. According to both Calipso and CC4CL, it is completely cloud covered. 

In case of single-layer cloudiness, CC4CL CTH agrees very well with Calipso (sector 1 and, in particular, sector 3). The CTH difference between CC4CL retrievals increases in the presence of overlapping clouds (sector 2). There are optically thin but vertically thick ($\approx$4 km) clouds in sector 2. For these the retrieved CTH is considerably underestimated by $\approx$6 km, which is probably a result of lower layer contributions that ``contaminate'' the satellite signals. Overall, about \load{phase_match_calipso_CC4CL_SIB_layer0}62.6 \% of CC4CL pixels agree with Calipso phase. Phase mismatch occurs in cases of single layer optically thin clouds (sector 2) and, less frequently, stratiform cloudiness (sector 3).

% Case 3 is located north of Siberia mainly covering the Kara and Barents Sea and crossing the Novaya Zemlya islands. For this case we have defined 3 sectors. The first one containing a vertically and horizontally somewhat inhomogeneous cloud field located between 4 and 7km. This layer is overlaid by very thin (COD lt 0.15) high-level cloud layers at the beginning and end of the sector, which are located at 11 and 10 km respectively. Sector 2 is dominated by a geometrically very thick, but optically very thin cloud layer with COT below 1, partly even below 0.15. The top of this layer is located around 10.5km and it has a geometrical thickness of about 4km. Underneath this layer some low-level and some mid-level clouds are found being strati-form organised. Sector 3 is very different, containing strati-form, single-layer cloud fields at 3km (first part of sector) and at 2km (second part of sector). These cloud fields have optical thicknesses larger than 1.

% The CC4CL cloud detection identifies all clouds correctly in case 3. In sector 1, CC4CL places the cloud tops around 5km, which is often close to the CALIOP result when neglecting the very thin, very high cloud layer in the first part of this sector. However, also some and optically thinner cloud layers near the mid-level cloud layer are missed. At the end of sector 1, CC4CL places some clouds further down in the troposphere, while there is some scatter between the passive sensors; in contrast to the consistent CC4CL CTP retrievals in the first part of the sector. 

% In sector 2, the CC4CL CTP is roughly around 4.5km, which is on the one hand in significant contrast to the geometrical cloud top height of the uppermost CALIOP clod layer (located around 10.5km), but on the other hand explainable since the upper cloud layer is optically thin. The measure signal seems again to be a mixture from that very high level, optically thin cloud layer and lower (partly optically thicker) layers of clouds or the surface. 

% The CC4CL phase shows some disagreement with CALIOP in the first part of the sector, showing liquid clouds (CALIOP has ice), which turns into a better agreement in the second part of the sector (now both have ice), once another cloud layer is underneath the first and not the surface. 

% In section 3 the agreement between CC4CL and CALIOP is remarkable, in the first and second part of the sector. CC4CL cloud top pressure are nearly exactly located at the top of the uppermost CALIOP cloud layer. An exception is here a slight overestimation of the CTH at the beginning of the sector under the presence of the optically very thin, high cloud layer. However, phase of CC4CL is in disagreement with CALIOP in the second part of the sector, when the clouds are higher, in which CC4CL determines ice and CALIOP liquid clouds.

\vspace{5mm}\underline{\textit{Case study AFR}}\vspace{2mm}

Study area AFR is located over the Gulf of Guinea and Western Africa, thus containing open ocean and snow/ice free land surfaces. As we excluded AATSR data, about 10 times more pixel collocations with Calipso are available (n = 1181) than for the Arctic cases. Additionally, measurements contain tropical and coastal cloud systems such as extensive low-level stratiform cloudiness and continental deep convection (\cref{fig:RGB_10241345}).

In general, the qualitative and qualitative agreement between CC4CL and Calipso CTH is remarkable. CC4CL data track the spatial pattern of continental CTH very well, which increases northwards and shows some small scale variability beyond 8\textdegree\ N. Moreover, the agreement in cloud coverage and phase is much better when compared to the Arctic scenes (values). However, CC4CL and the AVHRR heritage channel dataset are almost entirely insensitive to the very high, thin cloud layer in sector 2 and is rather driven by contributions from very low clouds or the sea surface. 

\subsubsection{Validation summary}

The three study areas show that CC4CL retrievals of CTP are very close to Calipso values for single layer, optically thick clouds. The mean CTP bias (CC4CL minus Calipso) is just \load{NA1_ctp_bias_part}2.2 hPa for NA1 and pixels at $>$ 73.7\textdegree\ N, and \load{SIB_ctp_bias_part}14.4 hPa for SIB at $>$ 74\textdegree\ N. In case of multi-layer clouds, CC4CL estimates are often located in between Calipso's top and bottom layer estimates, and rarely even below the lowest layer. For these cases, the optimal estimation algorithm processes satellite signals that are likely to contain radiance contributions from multiple cloud layers. The OE then optimizes the fit between modelled and observed radiances by placing the cloud lower in the atmospheric profile, and so the mixed nature of the satellite data leads to an overestimation of CTP. In general, underestimation of CTP is rare (\% of pixels whose CTP is CC4CL $<$ Calipso = \load{NA1_ctp_bias}30.6 (NA1), \load{NA2_ctp_bias}1.2 (NA2), and \load{SIB_ctp_bias}45.7 (SIB)).

There is no clear influence of the underlying land type or topography on retrieval values or the cloud mask. However, the limited sample size does not allow for generalizations. Only for site NA2, Calipso identified cloud-free pixels, \load{cfree_cc4cl_and_calipso_NA2}68.8\% of which were also detected as cloud-free by CC4CL's neural network cloud mask, and with few exceptions as low level water clouds otherwise. In few cases, CC4CL fails to detect clouds seen by Calipso (\% of missed clouds = \load{cfree_calipso_cloudy_CC4CL_NA1}47.2 (NA1), \load{cfree_calipso_cloudy_CC4CL_NA2}21.3 (NA2), \load{cfree_calipso_cloudy_CC4CL_SIB}47.2 (SIB)). We did not account for fractional cloud coverage, as we set a grid box as cloud covered if any corresponding CC4CL pixel contains cloud information. As a consequence, there are slightly more cloud covered pixels for the spatially higher resolved MODIS and AATSR data than AVHRR.

There is no Calipso cloud layer for which the comparison with CC4CL phase clearly agrees best. After having rounded CC4CL values to the nearest integer, the percentage of pixels with equal phase is lowest for the top layer at COD $>$ 0 (NA1 = \load{phase_match_calipso_CC4CL_NA1_layer0}49.4 \%, NA2 = \load{phase_match_calipso_CC4CL_NA2_layer0}32.2 \%, SIB = \load{phase_match_calipso_CC4CL_SIB_layer0}62.6 \%), but similar for the mid layer COD $>$ 0.15 (NA1 = \load{phase_match_calipso_CC4CL_NA1_layer1}52.6 \%, NA2 = \load{phase_match_calipso_CC4CL_NA2_layer1}46.5 \%, SIB = \load{phase_match_calipso_CC4CL_SIB_layer1}66.7 \%), and the bottom layer COD $>$ 1 (NA1 = \load{phase_match_calipso_CC4CL_NA1_layer2}46.7 \%, NA2 = \load{phase_match_calipso_CC4CL_NA2_layer2}57.4 \%, SIB = \load{phase_match_calipso_CC4CL_SIB_layer2}66.7 \%). When averaged over all layers, phase agreement is largest for site SIB (\load{phase_match_calipso_CC4CL_SIB_layermean}65.3 \%), and clearly lower for NA1 (\load{phase_match_calipso_CC4CL_NA1_layermean}49.6 \%) and NA2 (\load{phase_match_calipso_CC4CL_NA2_layermean}45.4 \%).

For ice clouds, the most frequently occuring cloud types are cirrus (ID=6) for Calipso and overlap (ID=8) or cirrus (ID=7) for CC4CL. Water cloud types are more heterogeneous and for Calipso predominantly low transparent (ID=0), but altostratus (ID=5) and altocumulus (ID=4) are also frequent. CC4CL water clouds are approximately equally distributed amongst water (ID=3) and supercooled (ID=4) cloud types. Statistics: percentages for most frequent cloud types. Also: Which cloud type is dominant if CC4CL phase is different to Calipso phase?

\begin{figure*}[h]
  %\centering
  \includegraphics[width=\textwidth]{figures/RGB_N18_01x01_07221915.png} %RGB_multi_01x01_07221915.png}
  \caption{Study area NA1 (North America 1). Red (Ch1), green (Ch2), blue (Ch4 - Ch5) image derived from NOAA18 data resampled to 0.01\textdegree$\times$0.01\textdegree\ resolution. Date of observation is 07/22/2008, 19.15 LST. Orange lines: extent of the collocated MODIS granule, yellow lines: extent of the collocated AATSR orbit, red line: Calipso track outside (dashed) and within (solid) study area.}
  \label{fig:RGB_07221915}
  \includegraphics[width=\textwidth]{figures/calipsoVsCci_07221915_nocot_uncorrectedCtp.png}
  \caption{Vertical cross section of study area NA1 (North America 1) along the Calipso track at 5 km horizontal resolution. Top: CTH for CC4CL retrievals (coloured points) and Calipso measurements (vertical bars), and surface elevation and surface type (blue = open water, green = land, grey = snow/ice). The Calipso data are shown for those pressure layers where the cumulative top-to-bottom COD exceeds a threshold value of 0 (top layer), 0.15 (mid layer), and 1 (bottom layer). Bottom: Cloud mask/phase (ice to water = red to blue, cloud free = white, not determined = grey) and type (see \cref{tab:cloud_types} for key/value pairs) for all three Calipso layers and CC4CL retrievals. For CC4CL, cloud phase was averaged when resampling, and cloud type was assigned to the most frequent class per grid box. Sectors of characteristic cloud fields are separated by black vertical lines. Number of pixels n = \load{NA1_length}120.}
  \label{fig:calipso_07221915}
\end{figure*}

\begin{figure*}[h]
  %\centering
  \includegraphics[width=\textwidth]{figures/RGB_N18_01x01_07222058.png}
  %\includegraphics[width=\textwidth]{figures/RGB_multi_01x01_07222058.png}
  \caption{Study area NA2 (North America 2). As \autoref{fig:RGB_07221915}, but at 07/22/2008, 20.58 LST.}
  \label{fig:RGB_07222058}
  \includegraphics[width=\textwidth]{figures/calipsoVsCci_07222058_nocot_uncorrectedCtp.png}
  \caption{Study area NA2 (North America 2). As \autoref{fig:calipso_07221915}, but at 07/22/2008, 20.58 LST. n = \load{NA2_length}163}
  \label{fig:calipso_07222058}
\end{figure*}

\begin{figure*}[h]
  %\centering
  \includegraphics[width=\textwidth]{figures/RGB_N18_01x01_07270810.png} %RGB_multi_01x01_07270810.png}
  \caption{Study area SIB (Siberia). As \autoref{fig:RGB_07221915}, but at 07/27/2008, 08.10 LST.}
  \label{fig:RGB_07270810}
  \includegraphics[width=\textwidth]{figures/calipsoVsCci_07270810_nocot_uncorrectedCtp.png}
  \caption{Study area SIB (Siberia). As \autoref{fig:calipso_07221915}, but at 07/27/2008, 08.10 LST. n = \load{SIB_length}116.}
  \label{fig:calipso_07271915}
\end{figure*}

\begin{figure*}[h]
  %\centering
  \includegraphics[width=\textwidth]{figures/RGB_N18_01x01_10241345.png} %RGB_multi_01x01_07270810.png}
  \caption{Study area AFR (Africa). As \autoref{fig:RGB_07221915}, but at 10/24/2009, 13.45 LST.}
  \label{fig:RGB_10241345}
  \includegraphics[width=\textwidth]{figures/calipsoVsCci_10241345_nocot_uncorrectedCtp.png}
  \caption{Study area AFR (Africa). As \autoref{fig:calipso_07221915}, but at 10/24/2009, 13.45 LST. Due to space restrictions, no cloud type values are shown in table. n = \load{AFR_length}1181.} 
  \label{fig:calipso_10241345}
\end{figure*}


% satellite orbits to be analysed:
% see file L2_samples/collocations.txt
% selection: $2008/07/22, UTC 19:15 - 19:19, Lat 70 - 73, Lon -109 - -112$


\conclusions[Discussion and conclusions]\label{conclusions}

\subsection{The flexibility of the optimal etimation approach}

%% - AVHRR and MODIS differences are smallest, AATSR deviates especially for CER (why?)
%% - differences between sensors are generally significant, with few exceptions
%% - general patterns reproduced by all sensors nonetheless, so data are comparable and replaceable (e.g. highly resolved MODIS Europe data)

In general, the retrieval values of all control vector variables agree qualitatively. The RGB images show that all major patterns of cloud coverage and structure are resolved by all three sensors. However, AATSR data show largest deviations to the other sensors (\cref{fig:histograms}). It is unlikely that differences in spectral response functions are the reason, as MODIS and AATSR heritage channels are relatively close in their spectral response but their retrieval values do differ considerably. MODIS and AVHRR disagree much more in their spectral response, causing a reflectance difference of up to 30--40 \% \citep{Trishchenko02}, but their retrieval values are much more similar nonetheless. The difference to AVHRR and MODIS is largest for CER, so microphysical processes appear to be most affected. CER and COT are derived from reflectance channels only. 

The differences between mean values are almost exclusively significant. Thus, from a statistical point of view, the samples we analysed for AVHRR, MODIS, and AATSR have been drawn from different populations and are thus inconsistent. However, differences in cloud conditions at the various observation times and sensor spatial resolution explain part of these discrepancies. Moreover, a non-significant t-Test result is possibly a too strict metric for estimating the comparability of retrieval results. There is a range of confounding processes that affect each individual retrieval estimate, such as observation times, spectral responses, calibration deficiencies, and a varying amount of cloudy pixels to be compared. The case studies clearly show that, under optimal conditions for single layer cloud retrievals, CC4CL products are consistent with Calipso and practically insensitive to sensor characteristics.

We suggest that AVHRR and MODIS data can be used interchangeably, depending on the user's application. AVHRR data provide long-term data recors from 1982, however at a relatively coarse resolution of 5 km $\times$ 3 km. The MODIS data record started in 2000, and is thus not long enough to construct cloud climatologies. However, L1 data are available at 1 km resolution, and orbit control is guaranteed. With CC4CL, we also produced 0.05\textdegree\ lat/lon daily composites for Europe (data not shown), which is close to MODIS's original resolution in that area. These data provide a more detailed view on cloud features that AVHRR does not provide. In that sense, CC4CL products retrieved from AVHRR and MODIS are complementary.

\subsection{The value of uncertainty quantification}

%% outcomes:
%% - uncertainty analysis reveals insights under which conditions the retrieval is less reliable
%% - CTP uncertainty clearly lower than CER and COT, which are still affected by bugs
%% - cloud mask uncertainty a valuable alternative measure

The retrieval uncertainties prove to be a valuable source of information. On the one hand, they are useful for several user applications, e.g. for model validation, data assimilation applications, or climate studies in general. On the other hand, they allow for diagnosis of potential retrieval shortcomes. For example, we see that COT uncertainty scales with COT itself and is thus heteroscedastic (see also \citet{Poulsen12}). CC4CL COT values are at times unnaturally large, and the associated uncertainty reflects that. Also, it shows under which conditions the optimal estimator does converge at a relatively large cost value. In the cases shown here, large uncertainties are associated with optically thick clouds or underlying snow/ice cover. COT and CER uncertainties are clearly largest, and reflect remaining shortcomings in retrieving these values. Possible explanations are errors in look-up-tables of forward model cloud properties and difficulties to retrieve optical properties for large solar zenith angles.

We applied an independent approach to quantify cloud mask uncertainty. It is a valuable information, as a neural network does not provide uncertainties of its output. The approach we adopted here is straightforward. When the NN output, which is a pseudo Calipso cloud optical depth, approaches a defined threshold value for cloudiness, the uncertainty increases towards a maximum of 50 \%. This maximum value expresses that a cloud mask value is basically random, as it is equally likely to be cloudy or cloud-free. With that in mind, the cloud mask uncertainty data are easy to interpret. For example, we see that sea-ice pixels classified as cloudy to the North of study area NA2 (\cref{fig:uncertainties}) show uncertainties of 40 - 50 \%. This indicates that the NN is sensitive to bright ground cover, which it might confuse with being clouds. We suggest that users of ESA Cloud\textunderscore cci data should frequently consult cloud mask uncertainties. If a conservative cloud mask is required, it can be easily built by setting a maximum value for an acceptable uncertainty level.

\subsection{Strengths and weaknesses}

The results clearly show that CC4CL retrieves CTH of single layer, optically thick clouds with high accuracy and precision. When compared to Calipso, the mean deviation for these cases is as low as 10--240 m CTH. This is a promising result, and shows that the general optimal estimation framework is robust and appropriate for retrieving cloud properties from AVHRR heritage channels. 

Except for the cloud mask, Calipso data are a source of validation with absolute independence from CC4CL. Moreover, they are considered to be the current benchmark of cloud vertical structure and phase \citep{Winker09}. The main limitation of CALIOP though is its nadir-only view, so that global coverage is very limited. Also, the instrument is only able to probe the full geometrical depth of clouds whose total optical thickness is not larger than probably 6--10 \citep{Karlsson10}. We found no clear relationship between CC4CL CTP uncertainty and the difference between CC4CL CTP and Calipso CTP (data not shown). This suggests that the AVHRR heritage channels provide independent information on cloud vertical structure that is not clearly related to Calipso's CTP estimates. Retrieval uncertainty is driven by background and forward model error as well as the mixed-signal satellite radiances rather than by the complex, real vertical cloud structure.

In case of multi-layer clouds, CC4CL is not able to retrieve CTH of the top layer. The estimate is rather a mid-layer cloud pressure value. This is an expected limitation of our framework. \citet{Poulsen12} found that ORAC CTP and CER estimates are robust when the top ice cloud layer is $>$ 5 optical depths, and otherwise are the weighted average of several cloud layers. The AVHRR heritage channels do not provide sufficient information on retrieving cloud vertical structure. In case of semi-transparent top layer clouds, the underground signal, may it stem from a cloud or the Earth's surface, contributes to the total TOA reflectance or brightness temperature to variable extent. This mixed satellite measurement is input to CC4CL, which retrieves cloud parameters assuming a single cloud layer. As brightness temperatures and reflectances of, e.g., a cold, semi-transparent, bright top layer cirrus cloud overlapping a warmer, opaque, and darker low level water cloud, are a mix of several contributing surfaces, so will be the final retrieval value. For CC4CL, we often see that the final CTH estimate is placed between top and lower levels. We are currently working on testing and implementing a multi-layer retrieval within CC4CL. However, this method requires additional MODIS channels other than those already containined within the AVHRR heritage set, and thus will not be applicable to a full AVHRR reprocessing. For ESA Cloud\textunderscore cci, the decision was made to deliberately trade spectral information provided by MODIS and AATSR for time series continuity. Thus, discontinuities due to changing spectral resolutions within the entire dataset are avoided (\textcolor{red}{cite Martins paper}). In addition, we introduced corrected estimates of CTP and the derived CTT and CTH. These account for the fact that retrieved CTP is structurally too low and not really representative of the cloud top. The correction is based on a vertical displacement of CTP along the atmospheric profile based on optical thickness and the cloud's extinction coefficient (\textcolor{red}{cite Greg's paper, where this is probably shown in detail}). The correction is only made for ice phase clouds.

On a first view, estimates of cloud phase appear reasonable when compared to Calipso. However, we find a best overall agreement of $\approx$ 65 \% for the lower layers (cumulative COD $>$ 0.15 and $>$ 1). This is just slightly better than a random guess of cloud phase, which would converge towards a 50 \% level of agreement with Calipso as the sample size increases. Cloud phase is generally difficult to estimate, and estimates of various satellite derived products disagree most for that variable (cite paper or Stefan's comparisons). An evaluation of MODIS Collection 6 cloud phase yielded a total cloud phase agreement of over 90 \% with CALIOP. However, as the study is exculsively based on single-phase cloudy pixels, the performance of MODIS C6 as applied to multi phase pixels is still unknown \citep{Marchant16}. Here, we also find very high scores for cloud phase determination if we restrict the analysis to optically thick, spatially extensive cloud fields such as in study site AFR. There, cloud phase agrees with lower layer CALIOP estimates by as much as 95 \%. 

The key problems for phase determination are vertical stratification and the lack of direct in-situ measurements of cloud phase. Calipso observations, and also DARDAR (cite), are currently considered to be the most advanced estimates of cloud phase, relying on active measurement principles \citep{Winker09,Karlsson10}. However, this assumption is primarily based on the physical theory underlying their retrievals, rather than on a comprehensive validation with independent observations of cloud phase. Within CC4CL, we apply the Pavolonis algorithm for phase detection \citep{Pavolonis05}. It was built on simulated radiance data for varying phase, and further adjusted after analysis with real satellite data. The algorithm itself is a decision tree that contains a set of fixed threshold values for input reflectances and brightness temperatures, and was tuned to AVHRR. Even though we expect differences in phase determination between AVHRR vs. MODIS and AATSR due to varying spectral response functions, these were not large for the three study sites (include statistic on mutual aggrement). \citet{Pavolonis05} state that their product could not be validated due to the lack of direct observations, but rather underwent a consistency check with ground-based, independent estimates. The overall relatively low degree of agreement with Calipso is not satisfying if Calipso was considered the truth. However, we refrain from concluding that the CC4CL phase estimate was unrealistic, as to date no robust in-situ observation is available. A further complication is the question for which CALIOP cloud layer a passive sensor retrieval is representative. For single layer, optically thick clouds, CC4CL can be compared with any layer exceeding a cumulative optical thickness of 0 or 1. If such a cloud layer was however covered by optically and geometrically think cirrus clouds, the satellite data are still dominated by lower cloud level reflectance and, in particular, emittance. Consequently, Pavolonis cloud phase is not a top layer estimate in such cases. For study area AFR, we also found situations where the NN cloud mask, which was trained with CALIOP data, correctly identifies thin high cirrus as cloud over ocean. . One potential improvement would be to use the NN to also provide an estimate of cloud phase. Initial tests indicated that this approach would indeed improve the (global) agreement with Calipso (cite Stefan or run tests), which is to be expected, as the NN is trained with Calipso data. However, no estimate of cloud type would be provided.



%% - single layer cloud retrievals most accurate
%% - multi layer cloud estimates of cloud top pressure are between-layer estimates
%% - phase determination not satisfying; how good is Calipso though?


%% \subsection{Robustness of CC4CL in its practical application}
%% \subsection{Limitations (all other than FM limitations, just touching on the latter)}
%% \subsection{Comparability of the datasets derived from different sensors (AVHRR, MODIS, AATSR)}
%% \subsection{Applicability for climate studies}


\begin{acknowledgements}
NASA kindly provided MODIS Collection 6 radiance data. ECMWF kindly provided support and assistance with their computing facilities for development and processing.
This research/work was supported by the European Space Agency through the Cloud\textunderscore cci project (contract No.: 4000109870/13/I-NB).

\end{acknowledgements}


\appendix

\section{ERA-Interim input fields}
\begin{table}[ht]
  \caption{ERA-Interim variables used within CC4CL. Variables marked with * are available at 0.1\textdegree spatial resolution, all others default to 0.72\textdegree.}
  \begin{tabular}{p{3.8cm}|l|r|p{1.5cm}}
    \hline
    variable name & abbrev. & ID & unit \\
    \hline
    \multicolumn{4}{c}{profile variables} \\
    \hline
    Geopotential & Z & 129 & m$^2$ s$^{-2}$ \\
    Temperature  & T & 130 & K \\
    Specific humidity & Q & 133 & kg kg$^{-1}$ \\
    Log. surface pressure & LNSP & 152 & Pa \\
    Ozone mass mixing ratio & O3 & 203 & kg kg$^{-1}$ \\
    \hline
    \multicolumn{4}{c}{surface and single level variables} \\
    \hline
    Sea-ice cover* & CI & 31 & (0-1) \\
    Snow albedo & ASN & 32 & (0-1) \\
    Sea surface temperature & SSTK & 34 & K \\
    Total column water vapour & TCWV & 137 & kg m$^{-2}$ \\
    Snow depth* & SD & 141 & m of water equivalent \\
    10 metre U wind component & U10M & 165 & m s$^{-1}$ \\
    10 metre U wind component & V10M & 166 & m s$^{-1}$ \\
    2 metre temperature & T2M & 167 & K \\
    Land/sea mask & LSM & 172 & (0,1) \\
    Skin temperature* & SKT & 235 & K \\
    \hline
  \end{tabular}
  \label{tab:ERA-Interim}
\end{table}

\section{Full list of variables in CC4CL Level-2 files}    %% Appendix
\begin{table}[ht]
  \caption{CC4CL L2 output variables. NN = neural network.}
  \begin{tabular}{l|l|l}
    \hline
    variable name & abbrev. & unit \\
    \hline
    \multicolumn{3}{c}{primary variables} \\
    \hline
    latitude & lat & degree \\
    longitude & lon & degree \\
    solar zenith & solzen & degree \\
    satellite zenith & satzen & degree \\
    relative azimuth & relaz & degree \\
    cloud top pressure & ctp & hPa \\
    cloud top height & cth & kilometer \\
    cloud top temperature & ctt & kelvin \\
    cloud liquid water path & cwp & g/m$^2$ \\
    cloud effective radius & cer & micrometer \\
    cloud optical thickness & cot & 1 \\
    NN cloud optical thickness & cccot & 1 \\
    cloud albedo & cla & 1 \\
    cloud effective emissivity & cee & 1 \\
    cloud fraction & cc\textunderscore total & 1 \\
    NN cloud mask & cldmask & 1 \\
    cloud phase flag & phflag & 1 \\
    Pavolonis cloud type & cldtype & 1 \\
    retrieval convergence flag & conv &  1 \\
    number of retrieval iterations & niter & 1 \\
    a priori cost at solution & costja & 1 \\
    measurement cost at solution & costjm & 1 \\
    quality control flag & qcflag & 1 \\
    land/sea flag & lsflag & 1 \\
    snow/ice mask & siflag & 1 \\
    illumination flag & ilflag & 1 \\
    surface temperature & stemp & kelvin \\
    \hline
    \multicolumn{3}{c}{secondary variables} \\
    \hline
    cloud optical thickness a priori & cot\textunderscore ap & 1 \\
    \hline
  \end{tabular}
  \label{tab:L2_variables_appendix}
\end{table}

\subsection{}                               %% Appendix A1, A2, etc.


\bibliographystyle{copernicus}
\bibliography{CC4CL}

%% Literature citations
%% command                        & example result
%% \citet{jones90}|               & Jones et al.\ (1990)
%% \citep{jones90}|               & (Jones et al., 1990)
%% \citep{jones90,jones93}|       & (Jones et al., 1990, 1993)
%% \citep[p.~32]{jones90}|        & (Jones et al., 1990, p.~32)
%% \citep[e.g.,][]{jones90}|      & (e.g., Jones et al., 1990)
%% \citep[e.g.,][p.~32]{jones90}| & (e.g., Jones et al., 1990, p.~32)
%% \citeauthor{jones90}|          & Jones et al.
%% \citeyear{jones90}|            & 1990






%% FIGURES %%%%%%%%%%%%%%%%%%%%%%%%%%%%%%%%%%%%%%%%%%%%%%%%%%%%%%%%%%%%%%%%%%%%


%% ONE-COLUMN FIGURES

%f
% \begin{figure}[t]
% \vspace*{2mm}
% \begin{center}
% %\includegraphics[width=8.3cm]{FILE NAME}
% \end{center}
% \caption{TEXT}
% \end{figure}



% %% TWO-COLUMN FIGURES

% %f
% \begin{figure*}[t]
% \vspace*{2mm}
% \begin{center}
% %\includegraphics[width=12cm]{FILE NAME}
% \end{center}
% \caption{TEXT}
% \end{figure*}


% %% TABLES %%%%%%%%%%%%%%%%%%%%%%%%%%%%%%%%%%%%%%%%%%%%%%%%%%%%%%%%%%%%%%%%%%%%


% %% ONE-COLUMN TABLE

% %t
% \begin{table}[t]
% \caption{TEXT}
% \vskip4mm
% \centering
% %\begin{tabular}{column = lcr}
% \begin{tabular}{lcr}
% \tophline

% \middlehline

% \bottomhline
% \end{tabular}
% \end{table}


% %% TWO-COLUMN TABLE

% %t
% \begin{table*}[t]
% \caption{TEXT}
% \vskip4mm
% \centering
% %\begin{tabular}{column = lcr}
% \begin{tabular}{lcr}
% \tophline

% \middlehline

% \bottomhline
% \end{tabular}
% \end{table*}


% %% The different columns must be seperated with a & command and should
% %% end with \\ to identify the column brake.

% %%%%%%%%%%%%%%%%%%%%%%%%%%%%%%%%%%%%%%%%%%%%%%%%%%%%%%%%%%%%%%%%%%%%%%%%%%%%%%


% %% If figures and tables must be numbered 1a, 1b, etc. the following command
% %% should be inserted before the begin{} command.

% \addtocounter{figure}{-1}\renewcommand{\thefigure}{\arabic{figure}a}


\end{document}
