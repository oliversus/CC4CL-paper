\documentclass[amt]{copernicus}

\usepackage{hyperref}
\usepackage[capitalise,noabbrev]{cleveref}
\usepackage{natbib}
\usepackage{url}
\usepackage{lineno}
\usepackage{graphicx}
\usepackage{url}
\usepackage{array}
\usepackage{color}
\newcolumntype{P}[1]{>{\centering\arraybackslash}p{#1}}
\newcommand{\load}[1]{}

\begin{document}

% \chardef\_=`_
\linenumbers

\title{The Community Cloud retrieval for CLimate (CC4CL). Part I: A
  framework applied to multiple satellite imaging sensors.}

\author[1]{Oliver Sus}
\author[1]{Martin Stengel}
\author[1]{Stefan Stapelberg}
\author[2]{Gregory McGarragh}
\author[3]{Caroline Poulsen}
\author[4]{Adam Povey}
\author[1]{Cornelia Schlundt}
\author[3]{Gareth Thomas}
\author[3]{Matthew Christensen}
\author[2]{Simon Proud}
\author[1]{Matthias Jerg}
\author[4]{Roy Grainger}
\author[1]{Rainer Hollmann}

\affil[1]{DWD - Deutscher Wetterdienst, Frankfurter Straße 135, 63067 Offenbach, Germany}
\affil[2]{Department of Physics, University of Oxford, Clarendon Laboratory, Parks Road, Oxford OX1 3PU, U.K.}
\affil[3]{RAL Space - Rutherford Appleton Laboratory, Chilton, Didcot, OX11 0QX, U.K.}
\affil[4]{National Centre for Earth Observation, Atmospheric, Oceanic and Planetary Physics, University of Oxford, Parks Road, Oxford OX1 3PU, U.K.}

\runningtitle{CC4CL}

\runningauthor{O. Sus et al.}

\correspondence{Martin Stengel\\ martin.stengel@dwd.de}

\received{}
\pubdiscuss{} %% only important for two-stage journals
\revised{}
\accepted{}
\published{}

\firstpage{1}

\maketitle
% \sloppy

CC4CL_part1_v0_OS.txt

% % ---------------------------------------------------------------------------
% outline
% 1) situation
%    - clouds in the climate system
%    - existing cloud property retrievals
%    - motivation of the Cloud CCI project
% 2) complication
%    - cloud remote sensing problems:
%      - overlap
%      - spatial extent vs. sensor resolution
%      - night-time vs day-time retrievals
%      - clouds over bright surfaces
%    - satellite data limitations:
%      - temporal coverage: AVHRR
%      - spatial resolution: MODIS, AATSR
%      - spectral resolution: MODIS, AATSR
%      - sounders (HIRS) and lidar (CALIOP on CALIPSO) vs imagers 
%      --> need to merge best of both worlds whilst accounting for
%          (inter-)calibration issues, orbital drift, changes in spectral resolution
%    - retrieval shortcomings
%      - availability of temporally consistent auxiliary datasets (e.g. BRDF,
%        atmospheric state, snow + sea-ice coverage)
%      - thresholding: simple, but no explicit radiative consistency between
%        observed radiances and retrieved cloud properties; not flexibly applicable
%        to new channels and/or sensors, no uncertainty quantification
%      - optimal estimation: complicated but sophisticated, accounting for
%        radiative consistency; however, local minimum problem; input background
%        uncertainties difficult to quantify; several OE-related assumptions
%        satisfied?
% 3) the CC4CL framework + key questions
%    - optimal estimation methodology applied as state-of-the-art approach within
%      cloud retrieval community
%    - the 4 advantages of CC4CL: consistency, simultaneity, uncertainty,
%      flexibility
%    - Are retrieval products as obtained from the various sensors consistent with
%      each other?
%    - How do CC4CL cloud property products compare to other existing retrieval
%      schemes and independent data sources?
% ---------------------------------------------------------------------------

\introduction

% 1) situation

%   clouds in the climate system 
Satellite data are an essential source of information in order to
better understand and predict climate change. Satellites provide global long-term
observations from which geophysical parameters can be derived. These data are used for
time-series analysis of climate variables, and also for the assimilation into
or validation of climate models \citep{Comiso14,Yang13}. A paramount goal of
these efforts is the comprehensive characterization of the global energy and
water budgets.

Clouds modify atmospheric windows and radiative forcings of major greenhouse
gases \citep{Kiehl97}, and thus considerably constrain the global energy
budget. However, clouds are difficult to quantify in terms of composition and
temporal or spatial distribution. Observations of passive imagers do not
sufficiently resolve several important cloud properties, such as
vertical structure, sub-pixel heterogeneity, and the physical rather than radiatively
effective cloud boundary. Moreover, several background conditions (state of
surface and atmosphere, viewing geometry, sensor calibration and spectral
response uncertainties) complicate cloud retrievals. These
complications propagate uncertainties into derived cloud properties
themselves \citep{Hamann14}. Nonetheless, passive satellite imagers are the
most widely used instruments for cloud retrievals, providing global coverage at
acceptable cost. 

%   Existing cloud property retrievals
Examples for satellites based climatologies exploiting these types of sensors are the
International Cloud Climatology Project (ISCCP) \citep{Rossow99}, the
Pathfinder Atmosphere Extended (PATMOS-x) dataset
\citep{Heidinger09,Heidinger12}, and the EUMETSAT Satellite Application
Facility on Climate Monitoring (CM SAF) cLoud, Albedo and RAdiation (CLARA-A1)
dataset \citep{Karlsson13}.
Satellite observations of clouds are available for the past 40 years. However,
the production of climatologies and trend analyses is a complicated
task. Data need to be carefully processed and analysed in order
to derive a consistent long-term data record from several intercalibrated satellite
platforms. Consistency can be traded for continuity, and multi-platform
algorithms could exploit additional data when newer sensors become
available. Modern sensors provide improved spectral and spatial resolutions,
and thus potentially better cloud retrievals. However, their data records are too short to produce
climatologies of $>$ 30 years, and discontinuities are built into time series
when higher resolution satellite data are input to the processing.
Major complications of cloud retrievals are optically transparent clouds, multi-layer or overlapping clouds, and effective cloud top height determination. The degree to which these complications can be addressed depends on the nature of the retrieval and the type of input satellite data used. MODIS provides a much larger spectral resolution than just the six AVHRR heritage channels. MODIS and atmospheric sounders are clearly superior when detecting cloud height through the application of the ``CO2-slicing'' technique. However, when consistent climatologies are to be built, time series length and spatiotemporal resolution limit the choice in retrieval type and input satelllite data.  

\newpage

%   motivation of the Cloud CCI project
The European Space Agency has established the ESA Climate
Change Initiative program \citep{ESA_CCI_web,Hollmann13} in order to tackle
the problems outlined above and to advance the knowledge of the climate system.
The project's primary focus is the production of
thirteen Essential Climate Variables (ECVs) for the three domains ocean,
atmosphere, and land. ECVs are being produced for various climate drivers such as ozone, sea surface
temperature, ice sheets, and clouds. This study has been financed as part of the cloud ECV component of
ESA CCI \citep{ESA_Cloud_CCI_web}.
The main objective of ESA Cloud\textunderscore cci is to develop a state-of-the-art open source community
cloud retrieval algorithm, which is capable of processing passive imager data for a
number of \mbox{(non-)European} satellites covering several decades.
We used satellite data as retrieval input from MODIS Aqua and Terra (2000--2014) \citep{King92}, AVHRR on NOAA-7 to
NOAA-19 and METOPA (1978--2014) \citep{Jacobowitz03}, and AATSR on ENVISAT (?--?).
Only the AVHRR-equivalent channels from MODIS and AATSR are
used, thus the resulting retrieval data are henceforth referred to as the ``AVHRR heritage
dataset''. Moreover, the
resulting time series are carefully validated against well established existing climatologies (ISCCP, PATMOS-x, CM SAF, and MODIS
Collection 6), reanalysis and model data (ERA-Interim and EC-Earth),
ground-truth synoptic observations, and CALIPSO Lidar data.

The CC4CL core algorithm was developed in a modular fashion
and provides open source access to support distribution and further development within the
scientific community. Particular attention was paid to provide the flexibility of processing multiple instruments
with one and the same framework, thus maximising the consistency of
cloud products independent of the sensor source. Across the solar and thermal spectrum, the framework
accounts for physical and radiative consistency amongst all output variables and with
input satellite radiances. One key novel feature is the production of uncertainty
estimates of retrieval parameters through explict error propagation from input to output data. With these criteria in mind, the Oxford and Rutherford Aerosol and
Cloud (ORAC) retrieval \citep{Thomas09, Poulsen12} was chosen out of three competing algorithms within a ``Round Robin'' selection process. 

In this study, we will present the key features of the CC4CL processing algorithm. We will particularly focus on discussing the novel features of the framework, which set it apart from other approaches: the optimal estimation approach in general, the explicit uncertainty quantification through rigorous propagation of all known error sources to the final product, and the consistency of our long-term, multi-platform timeseries provided on various resolutions, from 0.5\textdegree\ up to 0.02\textdegree. Through describing all key input data and processing steps, the future user will be informed about remaining deficiencies of this new dataset, and its potential applicability in climate studies. We will provide an overview of the retrieved and derived output variables. These will further be initially validated for a set of representative L2 output data scenes within a selected region against other retrieval algorithms and independent observations. 

Key questions, hypotheses, analyses:

Are the Cloud\textunderscore cci products derived from AVHRR NOAA18, MODIS AQUA, and AATSR comparable? Does CC4CL produce insignificant differences in retrieved variables despite of differences in spectral responses/LUTs? Analyse statistics of COT, CTP, cloud cover such as:
\begin{itemize}
\item histogram plot
\item define distribution type
\item mean, median, standard deviation, skewness, kurtosis
\item analyse differences for statistical significane (T-test)
\item if possible, analyse differences for bias and variance; plot residuals against CALIPSO variables to see if there are systematic differences
\item residual analysis
\item focus: we see more cloud fraction for MODIS in long term time series, is that confirmed in this scene? Possibly create new resolution data at 1km (MODIS orig.), 2km, 3km, 4km, 5km (AVHRR orig.) to see how cloud fraction changes. Refer to ``How small is a cloud'' paper.
\end{itemize}

Validation with CALIPSO: are there systematic biases? 
\begin{itemize}
\item analyse residuals between CALIPSO and Cloud\textunderscore cci variables, possibly also as a function of CTP or COT
\end{itemize}




% \section{Data and methods}\label{input_data}

\subsection{L1 satellite data}\label{sec:L1_data}

\subsubsection{AVHRR}

The Advanced Very High Resolution Radiometer (AVHRR) is a cross-track scanner with a 2900 km swath width, providing almost daily global coverage. The sensor is equipped with six spectral channels (\autoref{tab:channels}), out of which only five can be transmitted simultaneously so that either channel 3a or 3b is available. In-flight calibration is performed only for thermal channels with a stable blackbody and a space view as references. AVHRR was/is mounted on several NOAA platforms as well as on EUMETSAT's MetopA/B, all of which are sun-synchronous, polar orbiting satellites. Due to a lack of orbit control technology for all NOAA AVHRR's, there is considerable orbit drift in equatorial crossing times (ECT) both for morning (ECT $<$ 12:00 LST) and afternoon (ECT $>$ 12:00 LST) satellites. To reduce drift induced changes in retrieved cloud properties, any AVHRR is replaced with its corresponding morning or afternoon successor once available ($=$ the AVHRR prime record). Typically, one morning and one afternoon NOAA satellite are in orbit simultaneously. 

For CC4CL, we use Global Area Coverage (GAC) L1c data on a reduced spatial resolution of 1.1 km $\times$ 4 km at nadir \citep{Devasthale16}. The AVHRR GAC L1c data record, including advanced intercalibration efforts, was produced for ESA Cloud\textunderscore cci and CMSAF \citep{Schulz09,Karlsson13}. CC4CL processed AVHRR data from 08/1981 (NOAA-7) up to 12/2014 (MetopA + NOAA-19). We applied a filtering technique to noisy channel 3b data (cite), and a database algorithm for splitting midnight orbits and blacklisting. 

\begin{table}[h]
  \caption{The CC4CL AVHRR-heritage dataset channel characteristics for AVHRR, AATSR, and MODIS.}
  \begin{tabular}{l|p{1.5cm}p{1.5cm}p{1.8cm}}
    \hline
    & CC4CL channel ID & sensor channel ID & channel width ($\mu m$) \\
    \hline
    AVHRR & 1 & 1 & 0.58 -- 0.68 \\
          & 2 & 2 & 0.725 -- 1.10 \\
          & 3 & 3a & 1.58 -- 1.64 \\
          & 4 & 3b & 3.55 -- 3.93 \\
          & 5 & 4 & 10.50 -- 11.50 \\
          & 6 & 5 & 11.5 -- 12.5 \\
    \hline
    MODIS & 1 & 1 & 0.62 -- 0.67 \\
          & 2 & 2 & 0.841 -- 0.876 \\
          & 3 & 6 & 1.628 -- 1.652 \\
          & 4 & 20 & 3.66 -- 3.84 \\
          & 5 & 31 & 10.78 -- 11.28 \\
          & 6 & 32 & 11.77 -- 12.27 \\
    \hline
    AATSR & 1 & 1 & 0.545 -- 0.565 \\
          & 2 & 2 & 0.649 -- 0.669 \\
          & 3 & 4 & 1.58 -- 1.64 \\
          & 4 & 5 & 3.51 -- 3.89 \\
          & 5 & 6 & 10.4 -- 11.3 \\
          & 6 & 7 & 11.5 -- 12.5 \\
    \hline
  \end{tabular}
  \label{tab:channels}
\end{table}


\subsubsection{MODIS}

The Moderate Resolution Imaging Spectroradiometer (MODIS) is carried by NASA's Terra and Aqua satellite platforms in a near sun-synchronous polar orbit at 705 km altitude. Due to orbit control, ECT is constant at 10:30 LST for Terra, and 13:30 LST for Aqua. The Aqua satellite is a member of the ``A-Train'' constellation, which also includes the CALIPSO and CloudSat satellites. MODIS is a cross-track scanner with a 2330 km swath width, producing a complete near-global coverage in less than two days \citep{Xiong09}.

CC4CL is applied to Collection 6 MOD021km (Terra) and MYD021km (Aqua) L1b input data \citep{MODIS_L1B}. For the AVHRR-heritage dataset produced here, the NASA Goddard space flight center performed the spectral subsetting of the 36 MODIS channels available (see \autoref{tab:channels} for the channels extracted), and data were directly shipped to ECMWF for archiving. The files are stored in HDF-EOS format at 1km spatial resolution, with the 250 m and 500 m channels having been aggregated to 1 km resolution. MODIS L1b data are organized in granules, each of which contains \texttildelow 5 minutes of MODIS data or \texttildelow 203 scan lines. Geolocation information is provided in separate files for Terra (MOD03) and Aqua (MYD03), containing geodetic latitude and longitude and solar/satellite zenith and azimuth angles. L1b data are corrected for all known instrumental effects through on-board calibrator data, and are organized into a viewing swath matching the geolocation file structure \citep{MODIS_PUG}. With CC4CL, we processed data from 02/2000 (Terra) or 08/2002 (Aqua) to 12/2014. 


\subsubsection{ATSR}

The second and third generation Along Track Scanning Radiometers (ATSR-2 and Advanced ATSR, \citet{Merchant12}) were launched on ESA's polar orbiting satellites ERS-2 and ENVISAT in 04/1995 and 03/2002, respectively. Both platforms were put into a sun-synchronous orbit at \texttildelow 780 km altitude, with ECT $=$ 10:30 LST for ERS-2 and ECT $=$ 10:00 for ENVISAT. Both ATSRs are identical in their overall configuration except for data transfer bandwidth (\autoref{tab:channels}). ATSR is designed to be self calibrating, with two on-board black-body targets for calibration of the thermal channels, and a sun-illuminated opal visible calibration target for the visible/near-infrared channels. ATSR uses a dual view system: a nadir view, and a forward view scanning the surface at an angle of 55\textdegree. The continuous scanning pattern produces a nadir resolution of approximately 1 km $\times$ 1 km with a swath width of 512 pixels or \texttildelow 500 km, providing global coverage every six days. 

We used no forward view data for cloud retrievals, as the 3-dimensional cloud structure produces parallax effects which are not accounted for within the current forward model. With CC4CL, we processed ATSR data from (operational?) launch date until 05/2003 (ERS-2) and 04/2012 (ENVISAT). 

\subsection{Auxiliary data}

\subsubsection{ERA-Interim}\label{sec:ERA-Interim}

We use ERA-Interim data as an apriori and first-guess input for the optimal estimation retrieval, and as input for the neural network cloud mask. ERA-Interim is a reanalysis of the global atmosphere, and is available from 1979 until today \citep{ERAInterim,Dee11}. The atmospheric profile variables are defined at 60 vertical levels. The original horizontal resolution is defined through a T255 spherical-harmonic representation for the basic dynamical fields, and through a reduced Gaussian grid with \texttildelow 79 km spacing fur surface fields. We downloaded ERA-Interim data from the ECMWF's MARS archive at a default spatial resolution of 0.72\textdegree, and at 0.1\textdegree\ for the neural network cloud mask input variables (\autoref{tab:ERA-Interim}). We acquired analysis (i.e. not forecast) data at 6-hourly timesteps. After download, all files were converted from GRIB to NetCDF format and remapped to the CC4CL preprocessor grid through CDO. This was necessary, as ERA-Interim coordinates are defined at the cell boundaries, whereas they are defined at the cell centres within CC4CL. The reanalysis data are temporally interpolated onto the satellite image's center time. Two ERA-Interim files before and after this time are linearly weighted as a function of their relative time differences.

ERA-Interim's land-surface model still needs to be improved in terms of its simulation of soil hydrology and snow cover. This affects the utilization of satellite data over land surfaces within ERA-Interim, which has negative effects on the representation of clouds and precipitation \citep{ERAInterim}. The confidence in temperature trend estimates however has improved considerably, so that ERA-Interim data have been used as an alternative to observational datasets to monitor climate change \citep{Willett10}.


\begin{table}[h]
  \caption{ERA-Interim variables used within CC4CL. Variables marked with * are available at 0.1\textdegree spatial resolution, all others default to 0.72\textdegree.}
  \begin{tabular}{p{3.8cm}|l|r|p{1.5cm}}
    \hline
    variable name & abbrev. & ID & unit \\
    \hline
    \multicolumn{4}{c}{profile variables} \\
    \hline
    Geopotential & Z & 129 & m$^2$ s$^{-2}$ \\
    Temperature  & T & 130 & K \\
    Specific humidity & Q & 133 & kg kg$^{-1}$ \\
    Log. surface pressure & LNSP & 152 & Pa \\
    Ozone mass mixing ratio & O3 & 203 & kg kg$^{-1}$ \\
    \hline
    \multicolumn{4}{c}{surface and single level variables} \\
    \hline
    Sea-ice cover* & CI & 31 & (0-1) \\
    Snow albedo & ASN & 32 & (0-1) \\
    Sea surface temperature & SSTK & 34 & K \\
    Total column water vapour & TCWV & 137 & kg m$^{-2}$ \\
    Snow depth* & SD & 141 & m of water equivalent \\
    10 metre U wind component & U10M & 165 & m s$^{-1}$ \\
    10 metre U wind component & V10M & 166 & m s$^{-1}$ \\
    2 metre temperature & T2M & 167 & K \\
    Land/sea mask & LSM & 172 & (0,1) \\
    Skin temperature* & SKT & 235 & K \\
    \hline
  \end{tabular}
  \label{tab:ERA-Interim}
\end{table}

\subsubsection{Land use}\label{sec:USGS}

We downloaded USGS Land Use/Land Cover raster data from the global land cover characteristics database \citep{USGS}. The USGS data are used as a land sea mask within the optimal estimation retrieval, as well as a land cover classificator within the cloud mask and the Pavolonis cloud typing scheme (update on this?). The dataset is defined on a regular lat/lon grid with 0.05\textdegree\ resolution. The USGS land cover classification was primarily derived from 1 km AVHRR NDVI 10-day composites for April 1992 through March 1993 \citep{USGS}. 

\subsubsection{Land surface BRDF}\label{sec:BRDF}

MODIS Collection 6 BRDF data (MCD43C1, \citet{MODIS_BRDF}), providing kernel weights for the RossThickLiSparseReciprocal BRDF model, are used within the retrieval scheme to set surface albedo and bidirectional reflectance distribution conditions. These data are available every 8 days derived from cloud-cleared 16-day Terra and Aqua measurements, and provided in HDF-EOS format at 0.05\textdegree\ spatial resolution. MCD43C1 data are classified as high quality given sufficient observations, and otherwise a low quality estimate is produced based on climatology anisotropy models. Validation against albedo measurements made at Baseline Surface Radiation Network (BSRN) sites show that the black-sky and white-sky albedo computed from the single sensor MCD43A1 high quality product are well within 5\% of the measured albedo, while the low quality product is within 10\% (copied from Greg, citation?).

We regridded MCD43C1 data to instrument resolution through bilinear interpolation, and filled missing pixels within the time series with pixel values of the temporally closest 8-day composite file providing valid data. For the pre-MODIS era, we produced a BRDF climatology by averaging all data available for a particular time slot. MCD43C1 kernel weights are applied to all CC4CL sensors, assuming negligible differences in spectral response functions. 

\subsubsection{Land surface emissivity}\label{sec:emissivity}

For land surface emissivity, we used the CIMSS global land surface IR emissivity database created by the Baseline Fit method \citep{Seemann08}. These data are derived from the MODIS operational land surface emissivity product (MOD11), to which the fit method is applied for filling spectral gaps between channels. CIMSS emissivity data are available on a monthly basis at ten wavelengths with 0.05\textdegree\ spatial resolution.

As for BRDF, we produced a land surface emissivity climatology for the pre-MODIS era by averaging all data available for a particular month.



% \section{Processing chain}\label{processing_chain}

a. Pre-processing
i. Brief summary of auxiliary data (already addressed in paper I) and preprocessing grid structure
ii. Pavolonis phase determination
iii. NN cloud mask (brief)
(Description of RTTOV and a priori and first-guess values in ORAC paper)
b. Main processing
i. Refer to ORAC paper
c. Post-processing
i. Phase switching
ii. Final output variables


% \section{L2 data - analysis and initial validation}\label{L2_data}

The three study areas show that CC4CL retrievals of CTP are very close to Calipso values for single layer, optically thick clouds. For NA1 and pixels at $>$ 73.7\textdegree\ N, the CTP bias (CC4CL minus Calipso) is XY, and XY for SIB at $>$ 74\textdegree\ N. In case of multi-layer clouds, CC4CL estimates are often located in between Calipso's top and bottom layer estimates, and rarely even below the lowest layer. For these cases, the optimal estimation algorithm processes satellite signals that are likely to contain radiance contributions from multiple cloud layers. The OE then optimizes the fit between modelled and observed radiances by placing the cloud lower in the atmospheric profile, and so the mixed nature of the satellite data leads to an overestimation of CTP. Considerable underestimation of CTP was never observed (CTP bias min = \insertVariable{ctp_bias_na1}25.2 (NA1), \insertVariable{ctp_bias_na2}25.2 (NA2), and \insertVariable{ctp_bias_sib}25.2 (SIB) hPa).

There is no clear influence of the underlying land type or topography on retrieval values or the cloud mask. However, the limited sample size does not allow for generalizations. Only for site NA2, Calipso identified cloud-free pixels, XY\% of which were also detected as cloud-free by CC4CL's neural network cloud mask, and as low level water clouds otherwise. In very few cases (XY\%), CC4CL fails to correctly detect clouds. We did not account for fractional cloud coverage, as we set a grid box as cloud covered if any corresponding CC4CL pixel contains cloud information. As a consequence, there are slightly more cloud covered pixels for the spatially higher resolved MODIS and AATSR data than AVHRR.

Cloud phase classification agrees best when CC4CL values are compared with Calipso mid-level clouds, i.e. COT $>$ 0.15. For that layer, \insertVariable{phase_match}75.2\% of pixels agree in cloud phase with CC4CL phase having been rounded to the nearest integer. 

The minimum/mean/maximum time difference between MODIS AQUA and NOAA18 is:  3.35 min/ 4.76 min/ 5.5 min.

Need table for study area summary:
study areas: NA1, NA2, SIB
variables: 
   CTP bias (CC4CL minus Calipso 0, 1, 2) min, mean, max
   cloud mask: CC4CL, Calipso in \% of pixels per row
   liquid cloud fraction: CC4CL, Calipso in \% of pixels per row

\subsection{Validation with Calipso}

\begin{figure*}[h]
  %\centering
  \includegraphics[width=\textwidth]{figures/RGB_multi_01x01_07221915.png}
  \caption{Study area NA1 (North America 1). Red (Ch1), green (Ch2), blue (Ch4 - Ch5) image of NOAA18 (left), MODIS AQUA (centre), and AATSR (right) data resampled to 0.01\textdegree$\times$0.01\textdegree\ resolution. Date of observation is 07/22/2008, 19.15 LST.}
  \label{fig:RGB_07221915}
\end{figure*}

\begin{figure*}[h]
  %\centering
  \includegraphics[width=\textwidth]{figures/RGB_multi_01x01_07222058.png}
  \caption{Study area NA2 (North America 2). As \autoref{fig:RGB_07221915}, but at 07/22/2008, 20.58 LST.}
  \label{fig:RGB_07222058}
\end{figure*}

\begin{figure*}[h]
  %\centering
  \includegraphics[width=\textwidth]{figures/RGB_multi_01x01_07270810.png}
  \caption{Study area SIB (Siberia). As \autoref{fig:RGB_07221915}, but at 07/27/2008, 08.10 LST.}
  \label{fig:RGB_07270810}
\end{figure*}

\begin{figure*}[h]
  %\centering
  \includegraphics[width=\textwidth]{figures/calipsoVsCci_07221915_nocot_uncorrectedCtp.png}
  \caption{Study area NA1 (North America 1). Top: CTP for CC4CL retrievals (coloured points) and Calipso measurements (vertical bars), and surface elevation and surface type (blue = open water, green = land, grey = snow/ice). The Calipso data are shown for those pressure layers where the cumulative top-to-bottom COD exceeds a threshold value of 0 (top layer), 0.15 (mid layer), and 1 (bottom layer). Bottom: Cloud phase (ice to water = red to blue, cloud free = white, not determined = grey) and cloud type (add reference to cloud type table) for Calipso and CC4CL.}
  \label{fig:calipso_07221915}
\end{figure*}

\begin{figure*}[h]
  %\centering
  \includegraphics[width=\textwidth]{figures/calipsoVsCci_07222058_nocot_uncorrectedCtp.png}
  \caption{Study area NA2 (North America 2). As \autoref{fig:calipso_07221915}, but at 07/22/2008, 20.58 LST.}
  \label{fig:calipso_07222058}
\end{figure*}

\begin{figure*}[h]
  %\centering
  \includegraphics[width=\textwidth]{figures/calipsoVsCci_07270810_nocot_uncorrectedCtp.png}
  \caption{Study area SIB (Siberia). As \autoref{fig:calipso_07221915}, but at 07/27/2008, 08.10 LST.}
  \label{fig:calipso_07271915}
\end{figure*}


% satellite orbits to be analysed:
% see file L2_samples/collocations.txt
% selection: $2008/07/22, UTC 19:15 - 19:19, Lat 70 - 73, Lon -109 - -112$


% \conclusions[Discussion]\label{discussion}

\subsection{The flexibility of the optimal estimation approach}

In general, the retrieval values of all control vector variables for AVHRR, MODIS, and AATSR agree well enough to be used interchangeably. Absolute mean differences are small and $\leq$ 21.9 hPa for CTP, $\leq$ 1.3 for COT, and $\leq$ 2.1 for CER, and generally smaller than the mean retrieval uncertainties themselves \textcolor{red}{(calculate)}. Moreover, the RGB images show that all major patterns of cloud coverage and structure are resolved by all three sensors. However, AATSR data show largest deviations to the other sensors (\cref{fig:histograms}). It is unlikely that differences in spectral response functions are the reason. MODIS and AATSR heritage channels are relatively close in their spectral response but their retrieval values do differ considerably. Also, MODIS and AVHRR disagree much more in their spectral response, which results in a reflectance difference of up to 30--40 \% \citep{Trishchenko02}, but their retrieval values are much more similar nonetheless. The difference to AVHRR and MODIS is largest for CER, so microphysical variables, which are derived from reflectance data only, appear to be most affected. 

The differences between mean values are almost exclusively significant. Thus, from a statistical point of view, the samples we analysed for AVHRR, MODIS, and AATSR have been drawn from different populations and are thus statistically inconsistent. However, differences in cloud conditions at the various observation times and sensor spatial resolution explain part of these discrepancies. Moreover, a non-significant t-Test result is possibly a too strict metric for estimating the comparability of retrieval results. There is a range of confounding processes that affect each individual retrieval estimate, such as observation times, spectral responses, calibration deficiencies, and a varying amount of cloudy pixels to be compared. The case studies clearly show that, under optimal conditions for single layer cloud retrievals, CC4CL products are consistent with Calipso and practically insensitive to sensor characteristics.

We suggest that AVHRR and MODIS data can be used interchangeably, depending on the user's application. AVHRR data provide long-term data records from 1982, however at a relatively coarse resolution of 5 km $\times$ 3 km. The MODIS data record started in 2000, and is thus too short to be considered a climatology. However, L1 data are available at 1 km resolution, and orbit control is guaranteed. With CC4CL, we also produced 0.05\textdegree\ lat/lon daily composites for Europe (data not shown), which is close to MODIS's original resolution in that area. These data provide a more detailed view on cloud features than AVHRR. In that sense, CC4CL products retrieved from AVHRR and MODIS are complementary.

\subsection{The value of uncertainty quantification}

The retrieval uncertainties prove to be a valuable source of information. On the one hand, they are useful for several user applications, such as model validation, data assimilation applications, or climate studies in general. On the other hand, they allow for diagnosis of potential retrieval shortcomings. For example, we see that COT uncertainty scales with COT itself and is thus heteroscedastic (see also \citet{Poulsen12}). CC4CL COT values are at times unnaturally large, and the associated uncertainty reflects that. Also, it shows under which conditions the optimal estimator does converge at a relatively large cost value. In the cases shown here, large uncertainties are associated with optically thick clouds or underlying snow/ice cover. COT and CER uncertainties are clearly largest, and reflect remaining shortcomings in retrieving these values. Possible explanations are errors in look-up-tables of forward model cloud properties and difficulties to retrieve optical properties for large solar zenith angles.

We applied an independent approach to quantify cloud mask uncertainty. It is a valuable information, as a neural network does not provide output uncertainty. The approach we adopted here is straightforward. When the NN output, which is a pseudo Calipso cloud optical depth, approaches a defined threshold value for cloudiness, the uncertainty increases towards a maximum of 50 \%. This maximum value expresses that a cloud mask value is basically random, as it is equally likely to be cloudy or cloud-free. With that in mind, the cloud mask uncertainty data are easy to interpret. For example, we see that sea-ice pixels classified as cloudy to the North of study area NA2 (\cref{fig:uncertainties}) show uncertainties of 40 - 50 \%. This indicates that the NN is sensitive to bright ground cover, which it might confuse with clouds. We suggest that users of ESA Cloud\textunderscore cci data should routinely consult cloud mask uncertainties. If a more conservative cloud mask is required, it can be easily built by setting a maximum value for an acceptable uncertainty level.

\subsection{Strengths and weaknesses}

The results clearly show that CC4CL retrieves CTH of single layer, optically thick clouds with high accuracy and precision. When compared to Calipso, the mean deviation for these cases is as low as 10--240 m CTH. This is a promising result, and shows that the general optimal estimation framework is robust and appropriate for retrieving cloud properties from AVHRR heritage channels. 

In case of multi-layer clouds, CC4CL is not able to retrieve CTH of the top layer if the latter is optically thin. The estimate is rather a mid-layer cloud pressure value. This is an expected limitation of our framework, and also of other retrieval algorithms using passive sensor data \citep{Holz08,Karlsson10}. \citet{Poulsen12} found that ORAC CTP and CER estimates are robust when the top ice cloud layer is $>$ 5 optical depths, and otherwise are the weighted average of several cloud layers. The AVHRR heritage channels do not provide sufficient information on retrieving cloud vertical structure. In case of semi-transparent top layer clouds, the underground signal, may it stem from a cloud or the Earth's surface, contributes to the total TOA reflectance or brightness temperature to variable extent. This mixed satellite measurement is input to CC4CL, which retrieves cloud parameters assuming a single cloud layer. As brightness temperatures and reflectances of, e.g., a cold, semi-transparent, bright top layer cirrus cloud overlapping a warmer, opaque, and darker low level water cloud, are a mix of several contributing surfaces, so will be the final retrieval value. Any CTH retrieved from AVHRR (heritage) data is rather the radiatively efficient than physical or geometrical cloud top \citep{Karlsson13}. For CC4CL, we often see that the final CTH estimate is placed between top and lower levels and is thus an underestimate, which is a common problem amongst retrieval algorithms using passive sensor data \citep{Watts11,Holz08,Karlsson13}. 

Multi layer cloud property retrievals have been developed \citep{Watts11}, and we also implemented and tested such an approach within CC4CL. However, this method requires additional MODIS channels other than those already contained within the AVHRR heritage set, and thus will not be applicable to a full AVHRR reprocessing. For ESA Cloud\textunderscore cci, the decision was made to deliberately trade spectral information provided by MODIS and AATSR for time series continuity. Thus, discontinuities due to changing spectral resolutions within the entire dataset are avoided (\textcolor{red}{cite Martins paper}). In addition, we introduced corrected estimates of CTP and the derived CTT and CTH to get closer to the physical or geometrical cloud top. The correction is based on a vertical displacement of CTP along the atmospheric profile based on optical thickness and the cloud's extinction coefficient (\textcolor{red}{cite Greg's paper, where this is probably shown in detail}). The correction is only made for ice phase clouds.

On a first view, estimates of cloud phase appear reasonable when compared to Calipso. However, we find a best overall agreement of $\approx$ 65 \% for the lower layers (cumulative COD $>$ 0.15 or $>$ 1). This is just slightly better than a random guess of cloud phase, which would converge towards a 50 \% level of agreement with Calipso as the sample size increases. Cloud phase is generally difficult to quantify, and estimates of various satellite derived products disagree considerably for that variable \citep{Stengel15}. An evaluation of MODIS Collection 6 cloud phase yielded a total cloud phase agreement of over 90 \% with CALIOP. However, as the study is exclusively based on single-phase cloudy pixels, the performance of MODIS C6 as applied to multi phase pixels is still unknown \citep{Marchant16}. We also find very high scores for cloud phase determination if restricting the analysis to optically thick, spatially extensive cloud fields such as in study site AFR. There, cloud phase agrees with lower layer CALIOP estimates by as much as 95 \%. 

The key problems for phase determination are vertical stratification and the lack of direct in-situ measurements of cloud phase. Calipso observations, and also DARDAR \citep{Ceccaldi13}, are currently considered to be the most advanced estimates of cloud phase, relying on active measurement principles \citep{Winker09,Karlsson10}. However, this assumption is primarily based on the physical theory underlying their retrievals, rather than on a comprehensive validation with independent observations of cloud phase. 

Within CC4CL, we apply the Pavolonis algorithm for phase detection \citep{Pavolonis05}. It was built on simulated radiance data for varying phase, and further adjusted after analysis with real satellite data. The algorithm itself is a decision tree that contains a set of fixed threshold values for input reflectances and brightness temperatures, and was tuned to AVHRR. Even though we expect differences in phase determination between AVHRR vs. MODIS and AATSR due to varying spectral response functions, these were not large for the three study sites (\textcolor{red}{include statistic on mutual agreement}). \citet{Pavolonis05} state that their product could not be validated due to the lack of direct observations, but rather underwent a consistency check with ground-based, independent estimates. 

The overall relatively low degree of agreement between CC4CL and Calipso is not satisfying if Calipso was considered the truth. However, we refrain from concluding that the CC4CL phase estimate was unrealistic, as to date no robust, spatially resolved in-situ observations are available and our comparisons included multi-layered cloud conditions. It is difficult to determine the representative CALIOP cloud layer when validating a passive sensor retrieval. For single layer, optically thick clouds, CC4CL can be compared with any layer exceeding a cumulative optical thickness of 0 or 1. If such a cloud layer was however covered by optically and geometrically thin cirrus clouds, the satellite data are still dominated by lower cloud level reflectance and, in particular, emittance. Consequently, Pavolonis cloud phase is not a top layer estimate in such cases. For study area AFR, we also found situations where the NN cloud mask, which was trained with CALIOP data, correctly identifies thin high cirrus as cloud over ocean but the cloud type algorithm failed to identify its ice phase. One potential improvement would be to use the NN to also provide an estimate of cloud phase. Initial tests indicated that this approach would indeed improve the (global) agreement with Calipso (\textcolor{red}{cite Stefan or run tests}), which is to be expected, as the NN is trained with Calipso data. However, no estimate of cloud type would be provided.

Calipso data are considered to be the current benchmark of cloud detection, vertical structure, and phase \citep{Winker09,Karlsson13,Holz08}, and are -- except for the cloud mask -- a source of validation with absolute independence from CC4CL. The main limitation of CALIOP though is its nadir-only view, so that global coverage is very limited. Also, the instrument is only able to probe the full geometrical depth of clouds whose total optical thickness is not larger than probably 3--5 \citep{Karlsson13}. We found no clear relationship between CC4CL CTP uncertainty and the difference between CC4CL CTP and Calipso CTP (data not shown). This suggests that the AVHRR heritage channels provide independent information on cloud vertical structure that is not clearly related to Calipso's CTP estimates. Retrieval uncertainty is driven by background and forward model error as well as the mixed-signal satellite radiances rather than by the complex, real vertical cloud structure.


% \conclusions\label{conclusions}

We show that CC4CL is a robust and flexible framework for producing cloud products from passive satellite sensor data. Differences between retrieval values for collocated satellite data are negligible for AVHRR, MODIS, and AATSR. ESA Cloud\textunderscore cci data provide climatologies (AVHRR) as well as highly resolved snap-shots for selected regions (e.g. Europe, MODIS). The complete sensor set of CC4CL data forms a unique, coherent, long-term, multi-instrument cloud property product that exploits synergic capabilities of several EO missions. Compared to single sensor retrievals, CC4CL data are improved in terms of accuracy and spatiotemporal sampling.

CC4CL explicitly quantifies retrieval uncertainties according to rules of error propagation and optimal estimation theory. These uncertainties are a valuable source for model validation, data assimilation, climate studies, or retrieval diagnosis. Neural network cloud mask uncertainty is a novel feature that enables the user to assess product quality and to create individualized cloud masks.  %to build cloud masks that meet individual requirements.

We find that CC4CL is limited by weaknesses that are common to passive sensor cloud product retrievals. In general, an initial validation with CALIOP data shows that CTH of optically thin clouds is underestimated. In case of multi-layer clouds, the retrieved CTH is then rather a mix product of all radiatively contributing cloud layers. The AVHRR heritage channels do not provide sufficient physical information that would allow for detailed retrievals of cloud vertical structure. Moreover, the forward cloud model is structurally incomplete, as it assumes a single plane cloud layer. A multi-layer cloud property retrieval has been added to CC4CL, which however is only applicable to MODIS data. 

To account for CTH underestimation, we implemented a correction for CTH that assumes that passive sensor data see beyond the top into the clouds up to a penetration depth of $\approx$ 1 optical depth. Corrected cloud top values are stored as separate variables within CC4CL output files. 

Similarly, we find that the cloud phase estimate is only accurate for optimal retrieval conditions (optically thick top clouds). For a next reprocessing of the entire AVHRR data record, we replaced the \citet{Pavolonis05} algorithm with a neural network cloud phase estimation with better performance scores. 

Under optimal conditions for single layer cloud retrievals, CC4CL products show little sensitivity to sensor characteristics. Single layer, optically thick cloud retrievals are very accurate and precise when compared against CALIPSO (bias $<$ 240 m), which emphasizes the maturity and robustness of CC4CL. We thus recommend ESA Cloud\textunderscore cci data to be used for multi-annual studies of cloud parameters and more detailed assessments of regional patterns and diurnal variability.


% \begin{acknowledgements}
NASA kindly provided MODIS Collection 6 radiance data. ECMWF kindly provided support and assistance with their computing facilities for development and processing.
This research/work was supported by the European Space Agency through the Cloud\textunderscore cci project (contract No.: 4000109870/13/I-NB).

\end{acknowledgements}


% \input{appendix.tex}

\bibliographystyle{copernicus}
\bibliography{CC4CL}

%% Literature citations
%% command                        & example result
%% \citet{jones90}|               & Jones et al.\ (1990)
%% \citep{jones90}|               & (Jones et al., 1990)
%% \citep{jones90,jones93}|       & (Jones et al., 1990, 1993)
%% \citep[p.~32]{jones90}|        & (Jones et al., 1990, p.~32)
%% \citep[e.g.,][]{jones90}|      & (e.g., Jones et al., 1990)
%% \citep[e.g.,][p.~32]{jones90}| & (e.g., Jones et al., 1990, p.~32)
%% \citeauthor{jones90}|          & Jones et al.
%% \citeyear{jones90}|            & 1990


%% FIGURES %%%%%%%%%%%%%%%%%%%%%%%%%%%%%%%%%%%%%%%%%%%%%%%%%%%%%%%%%%%%%%%%%%%%


%% ONE-COLUMN FIGURES

% f
% \begin{figure}[t]
%   \vspace*{2mm}
%   \begin{center}
% %     \includegraphics[width=8.3cm]{FILE NAME}
%   \end{center}
%   \caption{TEXT}
% \end{figure}



% %% TWO-COLUMN FIGURES

% % f
% \begin{figure*}[t]
%   \vspace*{2mm}
%   \begin{center}
% %     \includegraphics[width=12cm]{FILE NAME}
%   \end{center}
%   \caption{TEXT}
% \end{figure*}


% %% TABLES %%%%%%%%%%%%%%%%%%%%%%%%%%%%%%%%%%%%%%%%%%%%%%%%%%%%%%%%%%%%%%%%%%%%


% %% ONE-COLUMN TABLE

% % t
% \begin{table}[t]
%   \caption{TEXT}
%   \vskip4mm
%   \centering
% %   \begin{tabular}{column = lcr}
%     \begin{tabular}{lcr}
%       \tophline

%       \middlehline

%       \bottomhline
%     \end{tabular}
%   \end{table}


% %%   TWO-COLUMN TABLE

% %   t
%   \begin{table*}[t]
%     \caption{TEXT}
%     \vskip4mm
%     \centering
% %     \begin{tabular}{column = lcr}
%       \begin{tabular}{lcr}
%         \tophline

%         \middlehline

%         \bottomhline
%       \end{tabular}
%     \end{table*}


% %%     The different columns must be seperated with a & command and should
% %%     end with \\ to identify the column brake.

% %%%%%%%%%%%%%%%%%%%%%%%%%%%%%%%%%%%%%%%%%%%%%%%%%%%%%%%%%%%%%%%%%%%%%%%%%%%%%%


% %%     If figures and tables must be numbered 1a, 1b, etc. the following command
% %%     should be inserted before the begin{} command.

%     \addtocounter{figure}{-1}\renewcommand{\thefigure}{\arabic{figure}a}


\end{document}
