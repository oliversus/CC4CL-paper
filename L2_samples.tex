\section{L2 data - analysis and initial validation}\label{L2_data}

The analysis and initial validation of L2 data is twofold. We first examine CC4CL cloud properties for one sample scene that extends approximately from 100\textdegree\ W to 170\textdegree\ W and 45\textdegree\ N to 75\textdegree\ N over North America. We focus on the consistency of retrieval values derived from different sensors (AVHRR, MODIS, AATSR). This also includes pixel-based uncertainties of the key variables ctp, cot, cer, and cldmask. We then perform an initial validation of retrieved cloud properties, for which CALIOP-CALIPSO data are our truth reference. This validation is limited to three high-latitude scenes for which collocations for all sensors with CALIOP are available.

% In this section we firstly show and discuss the cloud properties generated by CC4CL for one example scene. A particular focus here is the consistency among the sensors AVHRR, MODIS and AATSR. For the same scene we also present and discuss the pixel-based uncertainties of the core variables CMA, CTP, COT and CER. Secondly, a qualitative assessment of the retrieved cloud properties is carried out for three scenes using CALIOP-CALIPSO as reference.
% It needs to be noted that for matching the thee sensors mentioned above and CALIOP we are limited to the high-latitudes.
 
\subsection{CC4CL cloud properties}

The sample scene is characterized by various cloud types, and the CC4CL cloud mask defines only a relatively small fraction as cloud free (\cref{fig:CTP_intercomparison,fig:COT_intercomparison,fig:CER_intercomparison}).  When visually inspecting CC4CL output for CTP, COT, and CER, the retrieval data appear highly similar for the three different sensors. Main spatial patterns are resolved in all products. The data show that there are more cloud free AVHRR pixels, which is related to the coarser spatial resolution compared to MODIS and AATSR. The LST difference ranges from XY to XY minutes, and thus there is little cloud displacement between observation times.
 
CTP data are approximately normally distributed for all three sensors. Both COT and CER show positive kurtosis and skewness, as values close to 0 are most frequent. CER data are somewhat bimodal, having a primary peak at $\approx$12 micron and a secondary peak at $\approx$35 micron (\cref{fig:histograms,tab:retrieval_statistics}). Mean value differences are not significant between AVHRR and MODIS for CTP, MODIS and AATSR for COT, and AVHRR and AATSR for CER. The standard deviation of differences between two sensors are always lowest for AVHRR minus MODIS (\cref{tab:retrieval_statistics}). Significance tests of mean differences and standard deviations of residuals between sensor retrievals are sensitive to outliers. These are to some extent influenced by cloud displacement due to observation time differences. Even though we found no significant relationship between sensor retrieval residuals and observation time difference (data not shown), residuals are likely to be smaller and thus possibly insignificant if sensor obervation times were identical.
 
\subsection{Uncertainties}

Median absolute uncertainties are CTP = \load{NA2_ctp_unc_median}26.7 hPa, COT = \load{NA2_cot_unc_median}6.1, CER = \load{NA2_cer_unc_median}2.0 mic, and cloud mask = \load{NA2_cmask_unc_median}13.7 \% (\cref{fig:uncertainties}). The median relative retrieval uncertainty (not shown) is relatively low for all three retrieval variables (CTP = \load{NA2_ctp_unc_median_relative}4.7 \%, COT = \load{NA2_cot_unc_median_relative}6.1 \%, CER = \load{NA2_cer_unc_median_relative}2.0 \%). COT uncertainties increase with COT magnitude, and the RGB image (\cref{fig:RGB_07222058}) shows that largest uncertainties are found in cases of opaque cloud coverage and probably cloud over sea-ice surfaces. CER results are similar to COT, although relative uncertainties are somewhat lower. Cloud free areas show increased cloud mask uncertainties, and in particular over sea-ice surface areas. Note that the cloud mask uncertainties have been quantified as a function of the normalized difference to the cloud mask threshold, whereas relative retrieval uncertainties (100 $\times$ uncertainty $\div$ retrieved value) are shown for CTP, COT, and CER. 

\begin{table*}[t]
  \caption{Statistics of CTP, COT, and CER retrieval values for study area NA2 and AVHRR (first value in each cell), MODIS (second value), and AATSR (third value). $\Delta$ values are given for AVHRR minus MODIS (first value in each cell), AVHRR minus AATSR (second value), and MODIS minus AATSR (third value). $^{\ast}$t-Test p-value $>$ 0.1, indicating that differences in mean values are not significant.}
  \begin{tabular}{l|lllll}
    \hline
                 & mean & median & stddev & skewness & kurtosis \\
    \hline
             CTP & \load{ctpMeanN18}667.2, \load{ctpMeanMYD}665.0, \load{ctpMeanENV}645.2 & \load{ctpMedN18}667.8, \load{ctpMedMYD}668.1, \load{ctpMedENV}632.4 & \load{ctpStdN18}147.5, \load{ctpStdMYD}142.7, \load{ctpStdENV}146.2 & \load{ctpSkewN18}-0.2, \load{ctpSkewMYD}-0.2, \load{ctpSkewENV}0.1 & \load{ctpKurtN18}-0.4, \load{ctpKurtMYD}-0.4, \load{ctpKurtENV}-0.8 \\
    $\Delta$ CTP & \load{ctpdMeanN18}2.2$^{\ast}$, \load{ctpdMeanMYD}21.9, \load{ctpdMeanENV}19.7 & \load{ctpdMedN18}4.2, \load{ctpdMedMYD}22.3, \load{ctpdMedENV}18.5 & \load{ctpdStdN18}63.0, \load{ctpdStdMYD}138.7, \load{ctpdStdENV}138.9 & \load{ctpdSkewN18}-0.4, \load{ctpdSkewMYD}-0.3, \load{ctpdSkewENV}-0.3 & \load{ctpdKurtN18}8.2, \load{ctpdKurtMYD}1.0, \load{ctpdKurtENV}0.7 \\
             COT & \load{cotMeanN18}12.3, \load{cotMeanMYD}13.6, \load{cotMeanENV}13.4 & \load{cotMedN18}7.2, \load{cotMedMYD}8.6, \load{cotMedENV}8.8 & \load{cotStdN18}19.8, \load{cotStdMYD}19.7, \load{cotStdENV}17.6 & \load{cotSkewN18}6.6, \load{cotSkewMYD}5.7, \load{cotSkewENV}5.3 & \load{cotKurtN18}60.5, \load{cotKurtMYD}46.2, \load{cotKurtENV}40.8 \\
    $\Delta$ COT & \load{cotdMeanN18}-1.3, \load{cotdMeanMYD}-1.2, \load{cotdMeanENV}0.2$^{\ast}$ & \load{cotdMedN18}-0.6, \load{cotdMedMYD}-1.2, \load{cotdMedENV}-0.5 & \load{cotdStdN18}16.5, \load{cotdStdMYD}22.0, \load{cotdStdENV}21.3 & \load{cotdSkewN18}0.7, \load{cotdSkewMYD}2.4, \load{cotdSkewENV}1.8 & \load{cotdKurtN18}59.6, \load{cotdKurtMYD}41.5, \load{cotdKurtENV}33.1 \\
             CER & \load{cerMeanN18}21.1, \load{cerMeanMYD}19.2, \load{cerMeanENV}21.3 & \load{cerMedN18}16.5, \load{cerMedMYD}14.4, \load{cerMedENV}18.1 & \load{cerStdN18}13.0, \load{cerStdMYD}12.1, \load{cerStdENV}10.9 & \load{cerSkewN18}1.1, \load{cerSkewMYD}1.4, \load{cerSkewENV}0.6 & \load{cerKurtN18}1.4, \load{cerKurtMYD}1.2, \load{cerKurtENV}-0.8 \\
    $\Delta$ CER & \load{cerdMeanN18}1.9, \load{cerdMeanMYD}-0.2$^{\ast}$, \load{cerdMeanENV}-2.1 & \load{cerdMedN18}0.5, \load{cerdMedMYD}-1.0, \load{cerdMedENV}-1.9 & \load{cerdStdN18}7.0, \load{cerdStdMYD}11.6, \load{cerdStdENV}11.3 & \load{cerdSkewN18}0.8, \load{cerdSkewMYD}0.8, \load{cerdSkewENV}0.5 & \load{cerdKurtN18}7.9, \load{cerdKurtMYD}4.4, \load{cerdKurtENV}2.3 \\
    \hline
  \end{tabular}
  \label{tab:retrieval_statistics}
\end{table*}


\begin{figure*}[h]
  %\centering
  \includegraphics[width=\textwidth]{figures/07222058_ctp_multi.png}
  \caption{CTP retrieval values for study area NA2 with data from AVHRR (left), MODIS (middle), and AATSR (right).}
  \label{fig:CTP_intercomparison}
%\end{figure*}

%\begin{figure*}[h]
  %\centering
  \includegraphics[width=\textwidth]{figures/07222058_cot_multi.png}
  \caption{COT retrieval values for study area NA2 with data from AVHRR (left), MODIS (middle), and AATSR (right).}
  \label{fig:COT_intercomparison}
%\end{figure*}

%\begin{figure*}[h]
  %\centering
  \includegraphics[width=\textwidth]{figures/07222058_cer_multi.png}
  \caption{CER retrieval values for study area NA2 with data from AVHRR (left), MODIS (middle), and AATSR (right).}
  \label{fig:CER_intercomparison}
\end{figure*}

\begin{figure*}[h]
  %\centering
  \includegraphics[width=0.68\textwidth]{figures/07222058_uncertainties_absolute.png} %{figures/07222058_uncertainties_percent_goodspacing.png}
  \caption{Absolute uncertainties of MODIS AQUA retrieval data for study area NA2 and CTP [hPa], COT, CER [mic], and Cloud mask [\%].} %{Relative uncertainties [\%] of MODIS AQUA retrieval data for study area NA2 and CTP, COT, CER, and Cloud mask.}
  \label{fig:uncertainties}
\end{figure*}

\begin{figure*}[h]
  %\centering
  \includegraphics[width=\textwidth]{figures/07222058_histograms.png}
  \caption{Density histograms of NOAA18, MODIS AQUA, and AATSR retrieval data for study area NA2 and (a) CTP, (b) CTP differences, (c) COT, (d) COT differences, (e) CER, and (f) CER differences.}
  \label{fig:histograms}
\end{figure*}

\subsection{Validation with Calipso}

We found collocations between Calipso, AVHRR, MODIS, and AATSR for three study areas in the Arctic at 07/22/2008 19.15 LST (study area North America 1 = NA1, n = \load{NA1_length}120, \cref{fig:RGB_07221915}), 07/22/2008 20:58 LST (NA2, n = \load{NA2_length}163, \cref{fig:RGB_07222058}), and 07/27/2008 08.10 LST (Siberia = SIB, n = \load{SIB_length}116, \cref{fig:RGB_07270810}). These are located within 60\textdegree\ to 75\textdegree\ N latitude, and contain vegetated land, snow-covered land, open ocean, and sea-ice surfaces. For NA1 and SIB, all Calipso pixels were classified as cloud covered, but for NA2, about half of the pixels are cloud free.

When including AATSR, collocations are restricted to high latitude areas and the few pixels located at the intersection between Calipso and AATSR. We thus decided to extend the validation analysis by another scene without AATSR data, which we found in the Gulf of Guinea/West Africa between 7\textdegree\ S and 12\textdegree\ N at 24/10/2009 13.45 LST (Africa = AFR, n = \load{AFR_length}1181, \cref{fig:RGB_10241345}). For this scene, about ten times more pixel values are available than for the other scenes, displaying cloud systems such as low-level stratocumulus and deep convection that are not contained in the Arctic data shown here.

\subsubsection{Case studies}

We divided all study areas into logical sectors, for each of which a characteristic pattern of cloud coverage and type predominates. 

\vspace{5mm}\underline{\textit{Case study NA1}}\vspace{2mm}

Study area NA1 is a completely cloud covered scene over northern Canada containing land, ice-covered land, and open ocean surfaces. There is a mix of single and multi-layered clouds of variable optical thickness and height. 
CC4CL correctly classifies pixels as cloud covered, except for a few cases in sectors 3 and 4. CTH retrievals are very consistent among the three sensors, and only differ in sector 2. Clouds are generally retrieved too low compared to Calipso's top layer, unless the latter is optically thick as in sector 4. In the case of a (semi-)transparent cloud top layer, multiple surfaces contribute to the observed satellite data. CC4CL CTH is then located closer to (sector 1), at (sector 3), or even below the underlying cloud layer (sector 2). For single layer, optically thick (COT $>$ 1) cloud coverage however, CC4CL and Calipso CTH agree very well (sector 4). Under such conditions, the retrieval of CTP and derived products using AVHRR heritage channel data is very accurate. Cloud phase agreement between CC4CL and Calipso is very variable. It is best for optically thick high ice cloud coverage (sector 1), and worst for low water clouds (sector 4).

% For this scene we defined 4 sectors. In sector 1, a continuous, high cloud field with CTH around 9 km is found in CALIOP. This cloud layer consists of ice and is optically thin, i.e. the COT does not exceed 1, with a tendency of getting optically thinner towards the end of the sector. For some pixels a second cloud layer is found with CTH around 4 km, also mainly being ice. In addition some thin clouds are scattered a bit higher than the second cloud layer or (towards the end of the sector) in the lower troposphere. The latter being water clouds. In sector 2, the uppermost clouds are geometrically even thinner, with COT also not exceeding 1. In the last third of this sector, some optically thicker low-level clouds are present. Sector 3, similar to sector 1 shows an optically thin high cloud layer and a second layer 2 to 3 km below in which a total optical depth of 1 is exceeded. At the end of sector 3 the uppermost layer gets optically even thinner, while the second cloud layer disappears and a new second layer in the lower troposphere is present. Sector 4 is mainly characterized by single cloud layers, of optical thickness larger than 1, at 2km and 4km, respectively. 

% The CC4CL cloud detection identifies nearly all cloudy pixels for all three sensors. Only the end of sector 3 (all three sensors) and in sector 4 (only AVHRR) some few cloudy pixels are missed. Except in sector 2, the CC4CL CTP retrievals are mostly very consistent among the three sensors. However, the clouds are generally retrieved too low compared to the uppermost cloud layer top in CALIOP, if the uppermost layer is optically thin. In these conditions, the cloud top is usually placed between the optically thin top layer and the underlying layer with a tendency towards the lower layer. If no second cloud layer is present, the CTP is put close to the surface (e.g. in sector 2). In sector 3 the uppermost cloud layer is very thin, often below 0.15 COT), while together the second layer the COT reaches COT>1, leading to placing the CTP nearly directly at the vertical location of the second layer for all sensors.
% In contrast to sectors 1-3, in sector 4 the CC4CL CTP is in very good agreement with the location of the top of the uppermost CALIOP cloud layer. These layer are optically thicker (COT>1) and are not overlaid by thinner layers, supporting a very accurate CTP retrieval with CC4CL applied to the passive sensors.
% The agreement of the CC4CL phase with CALIOP is good in sector 1 until the presents of a very thin, high single layer cloud, where CC4CL switches to water phase while CALIOP has ice. The phase in sector 2 is very variable but some agreement of CC4CL to CALIOP can be found. In sector 3, the uppermost cloud layer is ice in CALIOP, to which the CC4CL phases mostly agree. Towards the second layer CALIOP gives liquid, while CC4CL mainly remains at the ice phase, however, with some liquid pixels inclusions. In sector 4 all clouds are liquid in CALIOP. CC4CL gives ice clouds for the second part of that sector, when the cloud top is near 4km. CTT retrievals show….K?! Generally speaking, the retrieved cloud phases among the passive sensors agree well.

\vspace{5mm}\underline{\textit{Case study NA2}}\vspace{2mm}

Study area NA2 is located entirely over snow/ice free land in Western Canada. Calipso cloud coverage is \load{cfree_calipso}47.2\%, spatially broken, and variable in height and phase. Clear sky pixels are mostly identified by CC4CL (\load{cfree_cc4cl_and_calipso_NA2}68.8\% correct), and cloudy pixels are rarely missed (\load{cloudy_cc4cl_and_calipso_NA2}78.7\% correct). CC4CL retrievals of thin high clouds and false positive cloudy pixels have low CTH values (sector 1). Calipso's small-scale horizontal variability in cloud phase is reflected by CC4CL data, which however overestimate the fraction of liquid water clouds in sector 2. CC4CL reproduces Calipso's spatial variability in CTH, which it only slightly underestimates by 0.5--1 km in sector 2. In sector 3 however, CC4CL considerably underestimates CTH by up to 7 km. Most of these clouds are optically and geometrically thin.

% Case 2 is located again over North America. First half (sector 1) of this case is mainly cloud free except some high, optically very thin clouds at 9,5km in the beginning. Sectors 2 and 3 contain mainly broken cloud fields that actually do show a lot of along-track variability. Clouds are vertically located around 4km in sector 2 and higher (mainly between 6 and 8km) in sector 3.

% Except from some scattered clouds in AATSR and MODIS, the CC4CL sensors generally reflect the clear-sky situations in sector 1. The high, very thin cloud layer at the beginning is missed. The obviously wrongly identified clouds in this sector are mostly low-level according to the CTP retrieval. In sector 2 all clouds are correctly identified, however, some small gaps seen in the CALIOP data are not identified by the CC4CL at all sensors. Only the AVHRR product does reflect them to some extend. The horizontal inhomogeneity in also found for all three passive sensors in terms of small-scale variabilities of cloud phase. 

% Generally speaking, CC4CL gives too often liquid phase in sector 2. The CC4CL CTP retrievals are very variable, as in CALIOP, with an underestimation of the cloud height of approx. 05. to 1km on average. The horizontal inhomogeneity in the cloud fields in sector 3 is similar to sector 2. The CC4CL results are similarly inhomogeneous. Even though most clouds are correctly identified by CC4CL in sector 3, the retrievals of phase and cloud top pressure are significantly deviating from CALIOP. The cloud top pressure difference between CALIOP and CC4CL partly amount to 7km.

\vspace{5mm}\underline{\textit{Case study SIB}}\vspace{2mm}

Study area SIB crosses the Novaya Zemlya islands north of Siberia and is defined by a mixture of open ocean and partially snow/ice covered land surfaces. According to both Calipso and CC4CL, it is completely cloud covered. 

In case of single-layer cloudiness, CC4CL CTH agrees very well with Calipso (sector 1 and, in particular, sector 3). The CTH difference between CC4CL retrievals increases in the presence of overlapping clouds (sector 2). There are optically thin but vertically thick ($\approx$4 km) clouds in sector 2. For these the retrieved CTH is considerably underestimated by $\approx$6 km, which is probably a result of lower layer contributions that ``contaminate'' the satellite signals. Overall, about \load{phase_match_calipso_CC4CL_SIB_layer0}62.6 \% of CC4CL pixels agree with Calipso phase. Phase mismatch occurs in cases of single layer optically thin clouds (sector 2) and, less frequently, stratiform cloudiness (sector 3).

% Case 3 is located north of Siberia mainly covering the Kara and Barents Sea and crossing the Novaya Zemlya islands. For this case we have defined 3 sectors. The first one containing a vertically and horizontally somewhat inhomogeneous cloud field located between 4 and 7km. This layer is overlaid by very thin (COD lt 0.15) high-level cloud layers at the beginning and end of the sector, which are located at 11 and 10 km respectively. Sector 2 is dominated by a geometrically very thick, but optically very thin cloud layer with COT below 1, partly even below 0.15. The top of this layer is located around 10.5km and it has a geometrical thickness of about 4km. Underneath this layer some low-level and some mid-level clouds are found being strati-form organised. Sector 3 is very different, containing strati-form, single-layer cloud fields at 3km (first part of sector) and at 2km (second part of sector). These cloud fields have optical thicknesses larger than 1.

% The CC4CL cloud detection identifies all clouds correctly in case 3. In sector 1, CC4CL places the cloud tops around 5km, which is often close to the CALIOP result when neglecting the very thin, very high cloud layer in the first part of this sector. However, also some and optically thinner cloud layers near the mid-level cloud layer are missed. At the end of sector 1, CC4CL places some clouds further down in the troposphere, while there is some scatter between the passive sensors; in contrast to the consistent CC4CL CTP retrievals in the first part of the sector. 

% In sector 2, the CC4CL CTP is roughly around 4.5km, which is on the one hand in significant contrast to the geometrical cloud top height of the uppermost CALIOP clod layer (located around 10.5km), but on the other hand explainable since the upper cloud layer is optically thin. The measure signal seems again to be a mixture from that very high level, optically thin cloud layer and lower (partly optically thicker) layers of clouds or the surface. 

% The CC4CL phase shows some disagreement with CALIOP in the first part of the sector, showing liquid clouds (CALIOP has ice), which turns into a better agreement in the second part of the sector (now both have ice), once another cloud layer is underneath the first and not the surface. 

% In section 3 the agreement between CC4CL and CALIOP is remarkable, in the first and second part of the sector. CC4CL cloud top pressure are nearly exactly located at the top of the uppermost CALIOP cloud layer. An exception is here a slight overestimation of the CTH at the beginning of the sector under the presence of the optically very thin, high cloud layer. However, phase of CC4CL is in disagreement with CALIOP in the second part of the sector, when the clouds are higher, in which CC4CL determines ice and CALIOP liquid clouds.

\vspace{5mm}\underline{\textit{Case study AFR}}\vspace{2mm}

Study area AFR is located over the Gulf of Guinea and Western Africa, thus containing open ocean and snow/ice free land surfaces. As we excluded AATSR data, about 10 times more pixel collocations with Calipso are available (n = 1181) than for the Arctic cases. Additionally, measurements contain tropical and coastal cloud systems such as extensive low-level stratiform cloudiness and continental deep convection (\cref{fig:RGB_10241345}).

In general, the quantitative and qualitative agreement between CC4CL and Calipso CTH is remarkable. CC4CL data track the spatial pattern of continental CTH very well (sector 3), which increases northwards and shows some small scale variability beyond 8\textdegree\ N. The height of the stratiform cloud field is almost identical for CC4CL and Calipso (sector 1). Only for the small layer located at 4 km height at $\approx$ 6.7\textdegree\ S, MODIS retrieval values differ somewhat from CC4CL AVHRR and Calipso.  Moreover, the agreement in cloud coverage and phase is much better when compared to the Arctic scenes (values). However, CC4CL and the AVHRR heritage channel dataset are almost entirely insensitive to the very high, thin cloud layer in sector 2. Here, CC4CL is rather driven by contributions from very low clouds or the sea surface. 

\subsubsection{Validation summary}

\textcolor{red}{Include AFR here, and show CTH instead of CTP values} The three study areas show that CC4CL retrievals of CTP are very close to Calipso values for single layer, optically thick clouds. The mean CTP bias (CC4CL minus Calipso) is just \load{NA1_ctp_bias_part}2.2 hPa for NA1 and pixels at $>$ 73.7\textdegree\ N, and \load{SIB_ctp_bias_part}14.4 hPa for SIB at $>$ 74\textdegree\ N. For multi-layer clouds, CC4CL estimates are often located in between Calipso's top and bottom layer estimates, but rarely below the lowest layer. For these cases, the optimal estimation algorithm processes satellite signals that are likely to contain radiance contributions from multiple cloud layers. The OE then optimizes the fit between modelled and observed radiances by placing the cloud lower in the atmospheric profile, and so the mixed nature of the satellite data leads to an overestimation of CTP. In general, underestimation of CTP is rare (\% of pixels whose CTP is CC4CL $<$ Calipso = \load{NA1_ctp_bias}30.6 (NA1), \load{NA2_ctp_bias}1.2 (NA2), and \load{SIB_ctp_bias}45.7 (SIB)).

There is no clear influence of the underlying land type or topography on retrieval values or the cloud mask. However, the limited sample size does not allow for generalizations. Only for site NA2, Calipso identified cloud-free pixels, \load{cfree_cc4cl_and_calipso_NA2}68.8\% of which were also detected as cloud-free by CC4CL's neural network cloud mask, and with few exceptions as low level water clouds otherwise. In some cases, CC4CL fails to detect clouds seen by Calipso (\% of missed clouds = \load{cfree_calipso_cloudy_CC4CL_NA1}47.2 (NA1), \load{cfree_calipso_cloudy_CC4CL_NA2}21.3 (NA2), \load{cfree_calipso_cloudy_CC4CL_SIB}47.2 (SIB) \textcolor{red}{these values seem wrong}). We did not account for fractional cloud coverage, as we set a grid box as cloud covered if any corresponding CC4CL pixel contains cloud information. As a consequence, there are slightly more cloud covered pixels for the spatially higher resolved MODIS and AATSR data than AVHRR.

There is no Calipso cloud layer for which the comparison with CC4CL phase clearly agrees best. After having rounded CC4CL values to the nearest integer, the percentage of pixels with equal phase is lowest for the top layer at COD $>$ 0 (NA1 = \load{phase_match_calipso_CC4CL_NA1_layer0}49.4 \%, NA2 = \load{phase_match_calipso_CC4CL_NA2_layer0}32.2 \%, SIB = \load{phase_match_calipso_CC4CL_SIB_layer0}62.6 \%), but similar for the mid layer COD $>$ 0.15 (NA1 = \load{phase_match_calipso_CC4CL_NA1_layer1}52.6 \%, NA2 = \load{phase_match_calipso_CC4CL_NA2_layer1}46.5 \%, SIB = \load{phase_match_calipso_CC4CL_SIB_layer1}66.7 \%), and the bottom layer COD $>$ 1 (NA1 = \load{phase_match_calipso_CC4CL_NA1_layer2}46.7 \%, NA2 = \load{phase_match_calipso_CC4CL_NA2_layer2}57.4 \%, SIB = \load{phase_match_calipso_CC4CL_SIB_layer2}66.7 \%). When averaged over all layers, phase agreement is largest for site SIB (\load{phase_match_calipso_CC4CL_SIB_layermean}65.3 \%), and clearly lower for NA1 (\load{phase_match_calipso_CC4CL_NA1_layermean}49.6 \%) and NA2 (\load{phase_match_calipso_CC4CL_NA2_layermean}45.4 \%).

For ice clouds, the most frequently occuring cloud types are cirrus (ID=6) for Calipso and overlap (ID=8) or cirrus (ID=7) for CC4CL. Water cloud types are more heterogeneous and for Calipso predominantly low transparent (ID=0), but altostratus (ID=5) and altocumulus (ID=4) are also frequent. CC4CL water clouds are approximately equally distributed amongst water (ID=3) and supercooled (ID=4) cloud types. Statistics: percentages for most frequent cloud types. Also: Which cloud type is dominant if CC4CL phase is different to Calipso phase?

\begin{figure*}[h]
  %\centering
  \includegraphics[width=\textwidth]{figures/RGB_N18_01x01_07221915.png} %RGB_multi_01x01_07221915.png}
  \caption{Study area NA1 (North America 1). Red (Ch1), green (Ch2), blue (Ch4 - Ch5) image derived from NOAA18 data resampled to 0.01\textdegree$\times$0.01\textdegree\ resolution. Date of observation is 07/22/2008, 19.15 LST. Orange lines: extent of the collocated MODIS granule, yellow lines: extent of the collocated AATSR orbit, red line: Calipso track outside (dashed) and within (solid) study area.}
  \label{fig:RGB_07221915}
  \includegraphics[width=\textwidth]{figures/calipsoVsCci_07221915_nocot_uncorrectedCtp.png}
  \caption{Vertical cross section of study area NA1 (North America 1) along the Calipso track at 5 km horizontal resolution. Top: CTH for CC4CL retrievals (coloured points) and Calipso measurements (vertical bars), and surface elevation and surface type (blue = open water, green = land, grey = snow/ice). The Calipso data are shown for those pressure layers where the cumulative top-to-bottom COD exceeds a threshold value of 0 (top layer), 0.15 (mid layer), and 1 (bottom layer). Bottom: Cloud mask/phase (ice to water = red to blue, cloud free = white, not determined = grey) and type (see \cref{tab:cloud_types} for key/value pairs) for all three Calipso layers and CC4CL retrievals. For CC4CL, cloud phase was averaged when resampling, and cloud type was assigned to the most frequent class per grid box. Sectors of characteristic cloud fields are separated by black vertical lines. Number of pixels n = \load{NA1_length}120.}
  \label{fig:calipso_07221915}
\end{figure*}

\begin{figure*}[h]
  %\centering
  \includegraphics[width=\textwidth]{figures/RGB_N18_01x01_07222058.png}
  %\includegraphics[width=\textwidth]{figures/RGB_multi_01x01_07222058.png}
  \caption{Study area NA2 (North America 2). As \autoref{fig:RGB_07221915}, but at 07/22/2008, 20.58 LST.}
  \label{fig:RGB_07222058}
  \includegraphics[width=\textwidth]{figures/calipsoVsCci_07222058_nocot_uncorrectedCtp.png}
  \caption{Study area NA2 (North America 2). As \autoref{fig:calipso_07221915}, but at 07/22/2008, 20.58 LST. n = \load{NA2_length}163}
  \label{fig:calipso_07222058}
\end{figure*}

\begin{figure*}[h]
  %\centering
  \includegraphics[width=\textwidth]{figures/RGB_N18_01x01_07270810.png} %RGB_multi_01x01_07270810.png}
  \caption{Study area SIB (Siberia). As \autoref{fig:RGB_07221915}, but at 07/27/2008, 08.10 LST.}
  \label{fig:RGB_07270810}
  \includegraphics[width=\textwidth]{figures/calipsoVsCci_07270810_nocot_uncorrectedCtp.png}
  \caption{Study area SIB (Siberia). As \autoref{fig:calipso_07221915}, but at 07/27/2008, 08.10 LST. n = \load{SIB_length}116.}
  \label{fig:calipso_07271915}
\end{figure*}

\begin{figure*}[h]
  %\centering
  \includegraphics[width=\textwidth]{figures/RGB_N18_01x01_10241345.png} %RGB_multi_01x01_07270810.png}
  \caption{Study area AFR (Africa). As \autoref{fig:RGB_07221915}, but at 10/24/2009, 13.45 LST.}
  \label{fig:RGB_10241345}
  \includegraphics[width=\textwidth]{figures/calipsoVsCci_10241345_nocot_uncorrectedCtp.png}
  \caption{Study area AFR (Africa). As \autoref{fig:calipso_07221915}, but at 10/24/2009, 13.45 LST. Due to space restrictions, no cloud type values are shown in table. n = \load{AFR_length}1181.} 
  \label{fig:calipso_10241345}
\end{figure*}


% satellite orbits to be analysed:
% see file L2_samples/collocations.txt
% selection: $2008/07/22, UTC 19:15 - 19:19, Lat 70 - 73, Lon -109 - -112$
