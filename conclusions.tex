\conclusions\label{conclusions}

We show that CC4CL is a robust and flexible framework for producing cloud products from passive satellite sensor data. Differences between retrieval values for collocated satellite data are negligible for AVHRR, MODIS, and AATSR. ESA Cloud\textunderscore cci data provide climatologies (AVHRR) as well as highly resolved snap-shots for selected regions (e.g. Europe, MODIS). The complete sensor set of CC4CL data forms a unique, coherent, long-term, multi-instrument cloud property product that exploits synergic capabilities of several EO missions. Compared to single sensor retrievals, CC4CL data are improved in terms of accuracy and spatiotemporal sampling.

CC4CL explicitly quantifies retrieval uncertainties according to rules of error propagation and optimal estimation theory. These uncertainties are a valuable source for model validation, data assimilation, climate studies, or retrieval diagnosis. Neural network cloud mask uncertainty is a novel feature that enables the user to assess product quality and to create individualized cloud masks.  %to build cloud masks that meet individual requirements.

We find that CC4CL is limited by weaknesses that are common to passive sensor cloud product retrievals. In general, an initial validation with CALIOP data shows that CTH of optically thin clouds is underestimated. In case of multi-layer clouds, the retrieved CTH is then rather a mix product of all radiatively contributing cloud layers. The AVHRR heritage channels do not provide sufficient physical information that would allow for detailed retrievals of cloud vertical structure. Moreover, the forward cloud model is structurally incomplete, as it assumes a single plane cloud layer. A multi-layer cloud property retrieval has been added to CC4CL, which however is only applicable to MODIS data. 

To account for CTH underestimation, we implemented a correction for CTH that assumes that passive sensor data see beyond the top into the clouds up to a penetration depth of $\approx$ 1 optical depth. Corrected cloud top values are stored as separate variables within CC4CL output files. 

Similarly, we find that the cloud phase estimate is only accurate for optimal retrieval conditions (optically thick top clouds). For a next reprocessing of the entire AVHRR data record, we replaced the \citet{Pavolonis05} algorithm with a neural network cloud phase estimation with better performance scores. 

Under optimal conditions for single layer cloud retrievals, CC4CL products show little sensitivity to sensor characteristics. Single layer, optically thick cloud retrievals are very accurate and precise when compared against CALIPSO (bias $<$ 240 m), which emphasizes the maturity and robustness of CC4CL. We thus recommend ESA Cloud\textunderscore cci data to be used for multi-annual studies of cloud parameters and more detailed assessments of regional patterns and diurnal variability.
