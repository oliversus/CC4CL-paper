\conclusions[Discussion and conclusions]\label{conclusions}

\subsection{The flexibility of the optimal etimation approach}

%% - AVHRR and MODIS differences are smallest, AATSR deviates especially for CER (why?)
%% - differences between sensors are generally significant, with few exceptions
%% - general patterns reproduced by all sensors nonetheless, so data are comparable and replaceable (e.g. highly resolved MODIS Europe data)

In general, the retrieval values of all control vector variables agree qualitatively. The RGB images show that all major patterns of cloud coverage and structure are resolved by all three sensors. However, AATSR data show largest deviations to the other sensors (\cref{fig:histograms}). It is unlikely that differences in spectral response functions are the reason, as MODIS and AATSR heritage channels are relatively close in their spectral response but their retrieval values do differ considerably. MODIS and AVHRR disagree much more in their spectral response, causing a reflectance difference of up to 30--40 \% \citep{Trishchenko02}, but their retrieval values are much more similar nonetheless. The difference to AVHRR and MODIS is largest for CER, so microphysical processes appear to be most affected. CER and COT are derived from reflectance channels only. 

The differences between mean values are almost exclusively significant. Thus, from a statistical point of view, the samples we analysed for AVHRR, MODIS, and AATSR have been drawn from different populations and are thus inconsistent. However, differences in cloud conditions at the various observation times and sensor spatial resolution explain part of these discrepancies. Moreover, a non-significant t-Test result is possibly a too strict metric for estimating the comparability of retrieval results. There is a range of confounding processes that affect each individual retrieval estimate, such as observation times, spectral responses, calibration deficiencies, and a varying amount of cloudy pixels to be compared. The case studies clearly show that, under optimal conditions for single layer cloud retrievals, CC4CL products are consistent with Calipso and practically insensitive to sensor characteristics.

We suggest that AVHRR and MODIS data can be used interchangeably, depending on the user's application. AVHRR data provide long-term data recors from 1982, however at a relatively coarse resolution of 5 km $\times$ 3 km. The MODIS data record started in 2000, and is thus not long enough to construct cloud climatologies. However, L1 data are available at 1 km resolution, and orbit control is guaranteed. With CC4CL, we also produced 0.05\textdegree\ lat/lon daily composites for Europe (data not shown), which is close to MODIS's original resolution in that area. These data provide a more detailed view on cloud features that AVHRR does not provide. In that sense, CC4CL products retrieved from AVHRR and MODIS are complementary.

\subsection{The value of uncertainty quantification}

%% outcomes:
%% - uncertainty analysis reveals insights under which conditions the retrieval is less reliable
%% - CTP uncertainty clearly lower than CER and COT, which are still affected by bugs
%% - cloud mask uncertainty a valuable alternative measure

The retrieval uncertainties prove to be a valuable source of information. On the one hand, they are useful for several user applications, e.g. for model validation, data assimilation applications, or climate studies in general. On the other hand, they allow for diagnosis of potential retrieval shortcomes. For example, we see that COT uncertainty scales with COT itself and is thus heteroscedastic (see also \citet{Poulsen12}). CC4CL COT values are at times unnaturally large, and the associated uncertainty reflects that. Also, it shows under which conditions the optimal estimator does converge at a relatively large cost value. In the cases shown here, large uncertainties are associated with optically thick clouds or underlying snow/ice cover. COT and CER uncertainties are clearly largest, and reflect remaining shortcomings in retrieving these values. Possible explanations are errors in look-up-tables of forward model cloud properties and difficulties to retrieve optical properties for large solar zenith angles.

We applied an independent approach to quantify cloud mask uncertainty. It is a valuable information, as a neural network does not provide uncertainties of its output. The approach we adopted here is straightforward. When the NN output, which is a pseudo Calipso cloud optical depth, approaches a defined threshold value for cloudiness, the uncertainty increases towards a maximum of 50 \%. This maximum value expresses that a cloud mask value is basically random, as it is equally likely to be cloudy or cloud-free. With that in mind, the cloud mask uncertainty data are easy to interpret. For example, we see that sea-ice pixels classified as cloudy to the North of study area NA2 (\cref{fig:uncertainties}) show uncertainties of 40 - 50 \%. This indicates that the NN is sensitive to bright ground cover, which it might confuse with being clouds. We suggest that users of ESA Cloud\textunderscore cci data should frequently consult cloud mask uncertainties. If a conservative cloud mask is required, it can be easily built by setting a maximum value for an acceptable uncertainty level.

\subsection{Strengths and weaknesses}

The results clearly show that CC4CL retrieves CTH of single layer, optically thick clouds with high accuracy and precision. When compared to Calipso, the mean deviation for these cases is as low as 10--240 m CTH. This is a promising result, and shows that the general optimal estimation framework is robust and appropriate for retrieving cloud properties from AVHRR heritage channels. 

Except for the cloud mask, Calipso data are a source of validation with absolute independence from CC4CL. Moreover, they are considered to be the current benchmark of cloud vertical structure and phase \citep{Winker09}. The main limitation of CALIOP though is its nadir-only view, so that global coverage is very limited. Also, the instrument is only able to probe the full geometrical depth of clouds whose total optical thickness is not larger than probably 6--10 \citep{Karlsson10}. We found no clear relationship between CC4CL CTP uncertainty and the difference between CC4CL CTP and Calipso CTP (data not shown). This suggests that the AVHRR heritage channels provide independent information on cloud vertical structure that is not clearly related to Calipso's CTP estimates. Retrieval uncertainty is driven by background and forward model error as well as the mixed-signal satellite radiances rather than by the complex, real vertical cloud structure.

In case of multi-layer clouds, CC4CL is not able to retrieve CTH of the top layer. The estimate is rather a mid-layer cloud pressure value. This is an expected limitation of our framework. \citet{Poulsen12} found that ORAC CTP and CER estimates are robust when the top ice cloud layer is $>$ 5 optical depths, and otherwise are the weighted average of several cloud layers. The AVHRR heritage channels do not provide sufficient information on retrieving cloud vertical structure. In case of semi-transparent top layer clouds, the underground signal, may it stem from a cloud or the Earth's surface, contributes to the total TOA reflectance or brightness temperature to variable extent. This mixed satellite measurement is input to CC4CL, which retrieves cloud parameters assuming a single cloud layer. As brightness temperatures and reflectances of, e.g., a cold, semi-transparent, bright top layer cirrus cloud overlapping a warmer, opaque, and darker low level water cloud, are a mix of several contributing surfaces, so will be the final retrieval value. For CC4CL, we often see that the final CTH estimate is placed between top and lower levels. We are currently working on testing and implementing a multi-layer retrieval within CC4CL. However, this method requires additional MODIS channels other than those already containined within the AVHRR heritage set, and thus will not be applicable to a full AVHRR reprocessing. For ESA Cloud\textunderscore cci, the decision was made to deliberately trade spectral information provided by MODIS and AATSR for time series continuity. Thus, discontinuities due to changing spectral resolutions within the entire dataset are avoided (\textcolor{red}{cite Martins paper}). In addition, we introduced corrected estimates of CTP and the derived CTT and CTH. These account for the fact that retrieved CTP is structurally too low and not really representative of the cloud top. The correction is based on a vertical displacement of CTP along the atmospheric profile based on optical thickness and the cloud's extinction coefficient (\textcolor{red}{cite Greg's paper, where this is probably shown in detail}). The correction is only made for ice phase clouds.

On a first view, estimates of cloud phase appear reasonable when compared to Calipso. However, we find a best overall agreement of $\approx$ 65 \% for the lower layers (cumulative COD $>$ 0.15 and $>$ 1). This is just slightly better than a random guess of cloud phase, which would converge towards a 50 \% level of agreement with Calipso as the sample size increases. Cloud phase is generally difficult to estimate, and estimates of various satellite derived products disagree most for that variable (cite paper or Stefan's comparisons). An evaluation of MODIS Collection 6 cloud phase yielded a total cloud phase agreement of over 90 \% with CALIOP. However, as the study is exculsively based on single-phase cloudy pixels, the performance of MODIS C6 as applied to multi phase pixels is still unknown \citep{Marchant16}. Here, we also find very high scores for cloud phase determination if we restrict the analysis to optically thick, spatially extensive cloud fields such as in study site AFR. There, cloud phase agrees with lower layer CALIOP estimates by as much as 95 \%. 

The key problems for phase determination are vertical stratification and the lack of direct in-situ measurements of cloud phase. Calipso observations, and also DARDAR (cite), are currently considered to be the most advanced estimates of cloud phase, relying on active measurement principles \citep{Winker09,Karlsson10}. However, this assumption is primarily based on the physical theory underlying their retrievals, rather than on a comprehensive validation with independent observations of cloud phase. Within CC4CL, we apply the Pavolonis algorithm for phase detection \citep{Pavolonis05}. It was built on simulated radiance data for varying phase, and further adjusted after analysis with real satellite data. The algorithm itself is a decision tree that contains a set of fixed threshold values for input reflectances and brightness temperatures, and was tuned to AVHRR. Even though we expect differences in phase determination between AVHRR vs. MODIS and AATSR due to varying spectral response functions, these were not large for the three study sites (include statistic on mutual aggrement). \citet{Pavolonis05} state that their product could not be validated due to the lack of direct observations, but rather underwent a consistency check with ground-based, independent estimates. The overall relatively low degree of agreement with Calipso is not satisfying if Calipso was considered the truth. However, we refrain from concluding that the CC4CL phase estimate was unrealistic, as to date no robust in-situ observation is available. A further complication is the question for which CALIOP cloud layer a passive sensor retrieval is representative. For single layer, optically thick clouds, CC4CL can be compared with any layer exceeding a cumulative optical thickness of 0 or 1. If such a cloud layer was however covered by optically and geometrically think cirrus clouds, the satellite data are still dominated by lower cloud level reflectance and, in particular, emittance. Consequently, Pavolonis cloud phase is not a top layer estimate in such cases. For study area AFR, we also found situations where the NN cloud mask, which was trained with CALIOP data, correctly identifies thin high cirrus as cloud over ocean. . One potential improvement would be to use the NN to also provide an estimate of cloud phase. Initial tests indicated that this approach would indeed improve the (global) agreement with Calipso (cite Stefan or run tests), which is to be expected, as the NN is trained with Calipso data. However, no estimate of cloud type would be provided.



%% - single layer cloud retrievals most accurate
%% - multi layer cloud estimates of cloud top pressure are between-layer estimates
%% - phase determination not satisfying; how good is Calipso though?


%% \subsection{Robustness of CC4CL in its practical application}
%% \subsection{Limitations (all other than FM limitations, just touching on the latter)}
%% \subsection{Comparability of the datasets derived from different sensors (AVHRR, MODIS, AATSR)}
%% \subsection{Applicability for climate studies}
