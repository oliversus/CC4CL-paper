\section{Data and methods}\label{input_data}

\subsection{L1 satellite data}\label{sec:L1_data}

\subsubsection{AVHRR}

The Advanced Very High Resolution Radiometer (AVHRR) is a cross-track scanner with a 2900 km swath width, providing almost daily global coverage. The sensor is equipped with six spectral channels (\autoref{tab:channels}), out of which only five can be transmitted simultaneously so that either channel 3a or 3b is available. In-flight calibration is performed only for thermal channels with a stable blackbody and a space view as references. AVHRR was/is mounted on several NOAA platforms as well as on EUMETSAT's MetopA/B, all of which are sun-synchronous, polar orbiting satellites. Due to a lack of orbit control technology for all NOAA AVHRR's, there is considerable orbit drift in equatorial crossing times (ECT) both for morning (ECT $<$ 12:00 LST) and afternoon (ECT $>$ 12:00 LST) satellites. To reduce drift induced changes in retrieved cloud properties, any AVHRR is replaced with its corresponding morning or afternoon successor once available ($=$ the AVHRR prime record). Typically, one morning and one afternoon NOAA satellite are in orbit simultaneously. 

For CC4CL, we use Global Area Coverage (GAC) L1c data on a reduced spatial resolution of 1.1 km $\times$ 4 km at nadir \citep{Devasthale16}. The AVHRR GAC L1c data record, including advanced inter-calibration efforts, was produced for ESA Cloud\textunderscore cci and CMSAF \citep{Schulz09,Karlsson13}. CC4CL processed AVHRR data from 08/1981 (NOAA-7) up to 12/2014 (MetopA + NOAA-19). We applied a filtering technique to noisy channel 3b data (cite), and a database algorithm for splitting midnight orbits and blacklisting. 

\begin{table}[h]
  \caption{The CC4CL AVHRR-heritage dataset channel characteristics for AVHRR, AATSR, and MODIS. Instrument noise is in reflectance for CC4CL channels 1-3, and in brightness temperature [K] for channels 4-6.}
  \begin{tabular}{l|p{0.7cm}p{0.7cm}p{1.8cm}p{0.5cm}}
    \hline
    & CC4CL ID & sensor ID & channel width ($\mu m$) & noise \\
    \hline
    AVHRR & 1 & 1 & 0.58 -- 0.68 & 0.005 \\
          & 2 & 2 & 0.725 -- 1.10 & 0.005 \\
          & 3 & 3a & 1.58 -- 1.64 & 0.005 \\
          & 4 & 3b & 3.55 -- 3.93 & 0.25 \\
          & 5 & 4 & 10.50 -- 11.50 & 0.2 \\
          & 6 & 5 & 11.5 -- 12.5 & 0.2 \\
    \hline
    MODIS & 1 & 1 & 0.62 -- 0.67 & 0.01 \\
          & 2 & 2 & 0.841 -- 0.876 & 0.01 \\
          & 3 & 6 & 1.628 -- 1.652 & 0.01 \\
          & 4 & 20 & 3.66 -- 3.84 & 0.2 \\
          & 5 & 31 & 10.78 -- 11.28 & 0.2 \\
          & 6 & 32 & 11.77 -- 12.27 & 0.2 \\
    \hline
    AATSR & 1 & 1 & 0.545 -- 0.565 & 0.005 \\
          & 2 & 2 & 0.649 -- 0.669 & 0.005 \\
          & 3 & 4 & 1.58 -- 1.64 & 0.005 \\
          & 4 & 5 & 3.51 -- 3.89 & 0.25 \\
          & 5 & 6 & 10.4 -- 11.3 & 0.1 \\
          & 6 & 7 & 11.5 -- 12.5 & 0.1 \\
    \hline
  \end{tabular}
  \label{tab:channels}
\end{table}

\subsubsection{MODIS}

The Moderate Resolution Imaging Spectroradiometer (MODIS) is carried by NASA's Terra and Aqua satellite platforms in a near sun-synchronous polar orbit at 705 km altitude. Due to orbit control, ECT is constant at 10:30 LST for Terra, and 13:30 LST for Aqua. The Aqua satellite is a member of the ``A-Train'' constellation, which also includes the CALIPSO and CloudSat satellites. MODIS is a cross-track scanner with a 2330 km swath width, producing a complete near-global coverage in less than two days \citep{Xiong09}.

CC4CL is applied to Collection 6 MOD021km (Terra) and MYD021km (Aqua) L1b input data \citep{MODIS_L1B}. For the AVHRR-heritage dataset produced here, the NASA Goddard space flight centre performed the spectral subsetting of the 36 MODIS channels available (see \autoref{tab:channels} for the channels extracted), and data were directly shipped to ECMWF for archiving. The files are stored in HDF-EOS format at 1km spatial resolution, with the 250 m and 500 m channels having been aggregated to 1 km resolution. MODIS L1b data are organized in granules, each of which contains \texttildelow 5 minutes of MODIS data or \texttildelow 203 scan lines. Geolocation information is provided in separate files for Terra (MOD03) and Aqua (MYD03), containing geodetic latitude and longitude and solar/satellite zenith and azimuth angles. L1b data are corrected for all known instrumental effects through on-board calibrator data, and are organized into a viewing swath matching the geolocation file structure \citep{MODIS_PUG}. With CC4CL, we processed data from 02/2000 (Terra) or 08/2002 (Aqua) to 12/2014. 

\subsubsection{ATSR-2 and AATSR}

The second and third generation Along Track Scanning Radiometers (ATSR-2 and Advanced ATSR, \citet{Merchant12}) were launched on ESA's polar orbiting satellites ERS-2 and ENVISAT in 04/1995 and 03/2002, respectively. Both platforms were put into a sun-synchronous orbit at \texttildelow 780 km altitude, with ECT $=$ 10:30 LST for ERS-2 and ECT $=$ 10:00 for ENVISAT. Both ATSRs are identical in their overall configuration except for data transfer bandwidth (\autoref{tab:channels}). ATSR is designed to be self calibrating, with two on-board black-body targets for calibration of the thermal channels, and a sun-illuminated opal visible calibration target for the visible/near-infrared channels. ATSR uses a dual view system: a nadir view, and a forward view scanning the surface at an angle of 55\textdegree. The continuous scanning pattern produces a nadir resolution of approximately 1 km $\times$ 1 km with a swath width of 512 pixels or \texttildelow 500 km, providing global coverage every six days. 

We used no forward view data for cloud retrievals, as the 3-dimensional cloud structure produces parallax effects which are not accounted for within the current forward model. With CC4CL, we processed ATSR-2 data from 08/1995 until 06/2002 and AATSR data from 07/2002 until 04/2012. 

\subsection{Auxiliary data}

\subsubsection{ERA-Interim}\label{sec:ERA-Interim}

We use ERA-Interim data as an apriori and first-guess input for the optimal estimation retrieval, and as input for the neural network cloud mask. ERA-Interim is a reanalysis of the global atmosphere, and is available from 1979 until today \citep{ERAInterim,Dee11}. The atmospheric profile variables are defined at 60 vertical levels. The original horizontal resolution is defined through a T255 spherical-harmonic representation for the basic dynamical fields, and through a reduced Gaussian grid with \texttildelow 79 km spacing for surface fields. We downloaded ERA-Interim data from the ECMWF's MARS archive at a default spatial resolution of 0.72\textdegree, and at 0.1\textdegree\ for the neural network cloud mask input variables (\autoref{tab:ERA-Interim}). We acquired analysis (i.e. not forecast) data at 6-hourly timesteps. After download, all files were converted from GRIB to NetCDF format and remapped to the CC4CL preprocessor grid through CDO. This was necessary, as ERA-Interim coordinates are defined at the cell boundaries, whereas they are defined at the cell centres within CC4CL. The reanalysis data are temporally interpolated onto the satellite image's centre time. Two ERA-Interim files before and after this time are linearly weighted as a function of their relative time differences.

ERA-Interim's land-surface model still needs to be improved in terms of its simulation of soil hydrology and snow cover. This affects the utilization of satellite data over land surfaces within ERA-Interim, which has negative effects on the representation of clouds and precipitation \citep{ERAInterim}. The confidence in temperature trend estimates however has improved considerably, so that ERA-Interim data have been used as an alternative to observational datasets to monitor climate change \citep{Willett10}.

\subsubsection{Land use}\label{sec:USGS}

We downloaded USGS Land Use/Land Cover raster data from the global land cover characteristics database \citep{USGS}. The USGS data are used as a land sea mask within the optimal estimation retrieval, as well as a land cover classificator within the cloud mask and the Pavolonis cloud typing scheme (update on this?). The dataset is defined on a regular lat/lon grid with 0.05\textdegree\ resolution. The USGS land cover classification was primarily derived from 1 km AVHRR NDVI 10-day composites for April 1992 through March 1993 \citep{USGS}. 

\subsubsection{Land surface BRDF}\label{sec:BRDF}

MODIS Collection 6 BRDF data (MCD43C1, \citet{MODIS_BRDF}), providing kernel weights for the Ross-Thick/Li-Sparse-Reciprocal BRDF model, are used within the retrieval scheme to set surface albedo and bidirectional reflectance distribution conditions. These data are available every 8 days derived from cloud-cleared 16-day Terra and Aqua measurements, and provided in HDF-EOS format at 0.05\textdegree\ spatial resolution. MCD43C1 data are classified as high quality given sufficient observations, and otherwise a low quality estimate is produced based on climatology anisotropy models. Validation against albedo measurements made at Baseline Surface Radiation Network (BSRN) sites show that the black-sky and white-sky albedo computed from the single sensor MCD43A1 high quality product are well within 5\% of the measured albedo, while the low quality product is within 10\% (copied from Greg, citation?).

We regridded MCD43C1 data to instrument resolution through bilinear interpolation, and filled missing pixels within the time series with pixel values of the temporally closest 8-day composite file providing valid data. For the pre-MODIS era, we produced a BRDF climatology by averaging all data available for a particular time slot. MCD43C1 kernel weights are applied to all CC4CL sensors, assuming negligible differences in spectral response functions. 

\subsubsection{Land surface emissivity}\label{sec:emissivity}

For land surface emissivity, we used the CIMSS global land surface IR emissivity database created by the Baseline Fit method \citep{Seemann08}. These data are derived from the MODIS operational land surface emissivity product (MOD11), to which the fit method is applied for filling spectral gaps between channels. CIMSS emissivity data are available on a monthly basis at ten wavelengths with 0.05\textdegree\ spatial resolution.

As for BRDF, we produced a land surface emissivity climatology for the pre-MODIS era by averaging all data available for a particular month.

\subsection{Collocating CC4CL L2 data and Calipso}

We resampled CC4CL L2 data to a regular latitude/longitude grid at 0.1\textdegree\ $\times$ 0.1\textdegree\ resolution (\textcolor{red}{check whether values for 0.05\textdegree are different}). This resampling is required for a intercomparison of CC4CL L2 data on a common grid, as differences in sensor spatial resolution are reduced when averaging all values available for each grid box. Calipso Level 2 5 km Cloud Layer data were produced by averaging over $\approx$14 samples beams with 70 m diameter taken every 335 m within a 5 km along-track corridor. Thus, Calipso data have a 70 m across-track $\times$ 5 km along-track spatial resolution (see also \citet{Holz08}), and the size of corresponding CC4CL grid box is approximately 11 km (meridional) $\times$ 2.9 to 5.6 km (zonal). As a consequence, the CC4CL grid boxes are larger than the reference Calipso pixels, but are still small enough to resolve some of the cloud features that Calipso observes. Note that AVHRR GAC data were produced by averaging 5 neighbouring pixels across-track, but Calipso data were averaged along-track.

