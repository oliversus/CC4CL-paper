% ---------------------------------------------------------------------------
% outline
% 1) situation
%    - clouds in the climate system
%    - existing cloud property retrievals
%    - motivation of the Cloud CCI project
% 2) complication
%    - cloud remote sensing problems:
%      - overlap
%      - spatial extent vs. sensor resolution
%      - night-time vs day-time retrievals
%      - clouds over bright surfaces
%    - satellite data limitations:
%      - temporal coverage: AVHRR
%      - spatial resolution: MODIS, AATSR
%      - spectral resolution: MODIS, AATSR
%      - sounders (HIRS) and lidar (CALIOP on CALIPSO) vs imagers 
%      --> need to merge best of both worlds whilst accounting for
%          (inter-)calibration issues, orbital drift, changes in spectral resolution
%    - retrieval shortcomings
%      - availability of temporally consistent auxiliary datasets (e.g. BRDF,
%        atmospheric state, snow + sea-ice coverage)
%      - thresholding: simple, but no explicit radiative consistency between
%        observed radiances and retrieved cloud properties; not flexibly applicable
%        to new channels and/or sensors, no uncertainty quantification
%      - optimal estimation: complicated but sophisticated, accounting for
%        radiative consistency; however, local minimum problem; input background
%        uncertainties difficult to quantify; several OE-related assumptions
%        satisfied?
% 3) the CC4CL framework + key questions
%    - optimal estimation methodology applied as state-of-the-art approach within
%      cloud retrieval community
%    - the 4 advantages of CC4CL: consistency, simultaneity, uncertainty,
%      flexibility
%    - Are retrieval products as obtained from the various sensors consistent with
%      each other?
%    - How do CC4CL cloud property products compare to other existing retrieval
%      schemes and independent data sources?
% ---------------------------------------------------------------------------

\introduction

% 1) situation

%   clouds in the climate system 
Satellite data are an essential source of information in order to
better understand and predict climate change. They provide global long-term
observations from which geophysical parameters can be derived. These data are used for
time-series analysis of climate variables, and also for the assimilation into
or validation of climate models \citep{Comiso14,Yang13}. A paramount goal of
these efforts is the comprehensive characterization of the global energy and
water budgets.

Clouds modify atmospheric windows and radiative forcings of major greenhouse
gases \citep{Kiehl97}, and thus considerably constrain the global energy
budget. However, clouds are difficult to quantify in terms of composition and
temporal or spatial distribution. Observations of passive imagers do not
sufficiently resolve several important cloud properties, such as
vertical structure, sub-pixel heterogeneity, and the physical rather than radiatively
effective cloud boundary. Moreover, several background conditions (state of
surface and atmosphere, viewing geometry, sensor calibration and spectral
response uncertainties) complicate cloud retrievals. These
complications propagate uncertainties into derived cloud properties
themselves \citep{Hamann14}. Nonetheless, passive satellite imagers are the
most widely used instruments for cloud retrievals, providing global coverage at
acceptable cost. 

%   Existing cloud property retrievals
Examples for satellites based climatologies exploiting these types of sensors are the
International Cloud Climatology Project (ISCCP) \citep{Rossow99}, the
Pathfinder Atmosphere Extended (PATMOS-x) dataset
\citep{Heidinger09,Heidinger12}, and the EUMETSAT Satellite Application
Facility on Climate Monitoring (CM SAF) cLoud, Albedo and RAdiation (CLARA-A1)
dataset \citep{Karlsson13}.
Satellite observations of clouds are available for the past 40 years. However,
the production of climatologies and trend analyses is a complicated
task. Data need to be carefully processed and analysed in order
to derive a consistent long-term data record from several intercalibrated satellite
platforms. Consistency can be traded for continuity, and multi-platform
algorithms could exploit additional data when newer sensors become
available. Modern sensors provide improved spectral and spatial resolutions,
and thus potentially better cloud retrievals. However, their data records are too short to produce
climatologies of $>$ 30 years, and discontinuities are built into time series
when higher resolution satellite data are input to the processing.
Major complications of cloud retrievals are optically transparent clouds, multi-layer or overlapping clouds, and effective cloud top height determination. The degree to which these complications can be addressed depends on the nature of the retrieval and the type of input satellite data used. MODIS provides a much larger spectral resolution than just the six AVHRR heritage channels. MODIS and atmospheric sounders are clearly superior when detecting cloud height through the application of the ``CO2-slicing'' technique. However, when consistent climatologies are to be built, time series length and spatiotemporal resolution limit the choice in retrieval type and input satelllite data.  

\newpage

%   motivation of the Cloud CCI project
The European Space Agency has established the ESA Climate
Change Initiative program \citep{ESA_CCI_web,Hollmann13} in order to tackle
the problems outlined above and to advance the knowledge of the climate system.
The project's primary focus is the production of
thirteen Essential Climate Variables (ECVs) for the three domains ocean,
atmosphere, and land. ECVs are being produced for various climate drivers such as ozone, sea surface
temperature, ice sheets, and clouds. This study has been financed as part of the cloud ECV component of
ESA CCI \citep{ESA_Cloud_CCI_web}.
The main objective of ESA Cloud\textunderscore cci is to develop a state-of-the-art open source community
cloud retrieval algorithm, which is capable of processing passive imager data for a
number of \mbox{(non-)European} satellites covering several decades.
We used satellite data as retrieval input from MODIS Aqua and Terra (2000--2014) \citep{King92}, AVHRR on NOAA-7 to
NOAA-19 and METOPA (1978--2014) \citep{Jacobowitz03}, and AATSR on ENVISAT (?--?).
Only the AVHRR-equivalent channels from MODIS and AATSR are
used, thus the resulting retrieval data are henceforth referred to as the ``AVHRR heritage
dataset''. Moreover, the
resulting time series are carefully validated against well established existing climatologies (ISCCP, PATMOS-x, CM SAF, and MODIS
Collection 6), reanalysis and model data (ERA-Interim and EC-Earth),
ground-truth synoptic observations, and CALIPSO Lidar data.

The CC4CL core algorithm was developed in a modular fashion
and provides open source access to support distribution and further development within the
scientific community. Particular attention was paid to provide the flexibility of processing multiple instruments
with one and the same framework, thus maximising the consistency of
cloud products independent of the sensor source. Across the solar and thermal spectrum, the framework
accounts for physical and radiative consistency amongst all output variables and with
input satellite radiances. One key novel feature is the production of uncertainty
estimates of retrieval parameters through explict error propagation from input to output data. With these criteria in mind, the Oxford and Rutherford Aerosol and
Cloud (ORAC) retrieval \citep{Thomas09, Poulsen12} was chosen out of three competing algorithms within a ``Round Robin'' selection process. 

In this study, we present the key features of the CC4CL processing algorithm. We particularly focus on discussing the novel features of the framework, which set it apart from other approaches: the optimal estimation approach in general, the explicit uncertainty quantification through rigorous propagation of all known error sources to the final product, and the consistency of our long-term, multi-platform timeseries provided on various resolutions, from 0.5\textdegree\ up to 0.02\textdegree. Through describing all key input data and processing steps, we inform the future user about all relevant features of this new dataset, and its potential applicability in climate studies. We provide an overview of the retrieved and derived output variables. These are initially validated in a comprehensive and detailed analysis of retrieval results that we collocated with Calipso observations for three scenes in the Arctic and one scene in the Gulf of Guinea/West Africa. The results show that CC4CL provides very realistic estimates of cloud top height and cover for optically thick clouds, but produces mixed-layer estimates for cases where optically thin clouds overlap. % for a set of representative L2 output data scenes within a selected region against other retrieval algorithms and independent observations. 

% Key questions, hypotheses, analyses:

% Are the Cloud\textunderscore cci products derived from AVHRR NOAA18, MODIS AQUA, and AATSR comparable? Does CC4CL produce insignificant differences in retrieved variables despite of differences in spectral responses/LUTs? Analyse statistics of COT, CTP, cloud cover such as:
% \begin{itemize}
% \item histogram plot
% \item define distribution type
% \item mean, median, standard deviation, skewness, kurtosis
% \item analyse differences for statistical significane (T-test)
% \item if possible, analyse differences for bias and variance; plot residuals against CALIPSO variables to see if there are systematic differences
% \item residual analysis
% \item focus: we see more cloud fraction for MODIS in long term time series, is that confirmed in this scene? Possibly create new resolution data at 1km (MODIS orig.), 2km, 3km, 4km, 5km (AVHRR orig.) to see how cloud fraction changes. Refer to ``How small is a cloud'' paper.
% \end{itemize}

% Validation with CALIPSO: are there systematic biases? 
% \begin{itemize}
% \item analyse residuals between CALIPSO and Cloud\textunderscore cci variables, possibly also as a function of CTP or COT
% \end{itemize}


% PPS und CLARA
% Round Robin paper: tabellarische Auflistung der Unterscheide zwischen ORAC und anderen.
% MODIS C6?
% generell: welche retrieval gibt es, und was sind die Hauptunterschiede? Bayesian, decision trees, ...
% Stärken und Schwächen von Retrievalverfahren, Vergleichsstudien
% Unüberwindbare Schwächen: keine vertikale Struktur, Integral von LWP/IWP wird eigentlich nicht gesehen

