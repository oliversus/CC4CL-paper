% ---------------------------------------------------------------------------
% outline
% 1) situation
%    - clouds in the climate system
%    - existing cloud property retrievals
%    - motivation of the Cloud CCI project
% 2) complication
%    - cloud remote sensing problems:
%      - overlap
%      - spatial extent vs. sensor resolution
%      - night-time vs day-time retrievals
%      - clouds over bright surfaces
%    - satellite data limitations:
%      - temporal coverage: AVHRR
%      - spatial resolution: MODIS, AATSR
%      - spectral resolution: MODIS, AATSR
%      - sounders (HIRS) and lidar (CALIOP on CALIPSO) vs imagers 
%      --> need to merge best of both worlds whilst accounting for
%          (inter-)calibration issues, orbital drift, changes in spectral resolution
%    - retrieval shortcomings
%      - availability of temporally consistent auxiliary datasets (e.g. BRDF,
%        atmospheric state, snow + sea-ice coverage)
%      - thresholding: simple, but no explicit radiative consistency between
%        observed radiances and retrieved cloud properties; not flexibly applicable
%        to new channels and/or sensors, no uncertainty quantification
%      - optimal estimation: complicated but sophisticated, accounting for
%        radiative consistency; however, local minimum problem; input background
%        uncertainties difficult to quantify; several OE-related assumptions
%        satisfied?
% 3) the CC4CL framework + key questions
%    - optimal estimation methodology applied as state-of-the-art approach within
%      cloud retrieval community
%    - the 4 advantages of CC4CL: consistency, simultaneity, uncertainty,
%      flexibility
%    - Are retrieval products as obtained from the various sensors consistent with
%      each other?
%    - How do CC4CL cloud property products compare to other existing retrieval
%      schemes and independent data sources?
% ---------------------------------------------------------------------------

\introduction

% 1) situation

%   clouds in the climate system 
Satellite data are an essential source of information in order to
better understand and predict climate change. Satellites provide global long-term
observations from which geophysical parameters can be derived. These data are used for
time-series analysis of climate variables, and also for the assimilation into
or validation of climate models \citep{Comiso14,Yang13}. A paramount goal of
these efforts is the comprehensive characterization of the global energy and
water budgets.

Clouds modify atmospheric windows and radiative forcings of major greenhouse
gases \citep{Kiehl97}, and thus considerably constrain the global energy
budget. However, clouds are difficult to quantify in terms of composition and
temporal or spatial distribution. Observations of passive imagers do not
sufficiently resolve several important cloud properties, such as
vertical structure, sub-pixel heterogeneity, and the physical rather than radiatively
effective cloud boundary. Moreover, several background conditions (state of
surface and atmosphere, viewing geometry, sensor calibration and spectral
response uncertainties) complicate cloud retrievals in various ways. These
complications propagate uncertainties into retrieved cloud properties
themselves \citep{Hamann14}. Nonetheless, passive satellite imagers are the
most widely used and available instruments, providing global coverage at
acceptable cost. 

%   Existing cloud property retrievals
Examples for satellites based climatologies exploiting these types of sensors are the
International Cloud Climatology Project (ISCCP) \citep{Rossow99}, the
Pathfinder Atmosphere Extended (PATMOS-x) dataset
\citep{Heidinger09,Heidinger12}, and the EUMETSAT Satellite Application
Facility on Climate Monitoring (CM SAF) cLoud, Albedo and RAdiation (CLARA-A1)
dataset \citep{Karlsson13}.
Thus, satellite observations of clouds have
been made for about 40 years now, yet many long-term observations and the
deduction of trends are complicated by a lack of consistency in the derivation
and treatment of the data at various processing and analysis stages.

%   motivation of the Cloud CCI project
In order to tackle the problems outlined above and to advance the knowledge of
the climate system, the European Space Agency has established the ESA Climate
Change Initiative program \citep{ESA_CCI_web,Hollmann13}.
Its purpose in the present initial phase is to focus on
thirteen Essential Climate Variables (ECVs) out of the three domains ocean,
atmosphere and land.
Among those ECVs are for example ozone, sea surface
temperature, ice sheets and clouds. 
The presented work has been carried out as part of the cloud ECV component  of
ESA's Climate Change Initiative program \citep{ESA_Cloud_CCI_web}.
The projects main objective is to develop a state-of-the-art open source community
cloud retrieval algorithm capable of processing passive imager data of a
number of European and Non-European satellites covering several years.
Specifically, the used input data span the years 2007 to 2009 of the sensors
MODIS on the Aqua and Terra satellites \citep{King92}, AVHRR on NOAA-15 to
NOAA-18 \citep{Jacobowitz03} and AATSR on ENVISAT.
However, only the AVHRR-equivalent channels from MODIS and AATSR are
used.
The resulting retrieval data are therefore termed ``AVHRR heritage
dataset''. 
Moreover, the project comprises a validation component in which the
resulting time series is carefully investigated and compared not only with
well established existing climatologies like ISCCP, PATMOS-x, CM SAF and MODIS
as well as reanalysis and model data like ERA-Interim and EC-Earth but also
where sensor based results are compared directly with A-train
observations. 
The development and application of the algorithm is governed by a number of
cornerstones.
Those comprise its open source and modular design character and
availability in order to enable a later release to and development within the
scientific community.
Moreover, with regards to its scientific capabilities it
is required to process multiple instruments in a consistent and transparent
manner, to utilize the provided input coherently and provide output equally
physically and radiatively coherent across the solar and thermal spectrum,
consistent and simultaneously for all output variables.
Moreover, it is required to provide associated uncertainty estimates for the
primary retrieval parameters through rigorous mathematical utilization of the
input data and error propagation from the input to the output data.
%     Round robin (to 3)
In order to investigate different algorithms for their potential usage and
applicability in light of the aforementioned criteria, a rigorous and
comprehensive comparison effort was carried out at the onset of the
project.
Taking part in this investigation, termed ``Round Robin Exercise'',
were three different algorithms. The operational processing system of the
\citet{CMSAF_web}, the CLAVR-X algorithm used to generate the PATMOS-x
climatology \citep{Heidinger13}, and the Oxford and Rutherford Aerosol and
Cloud (ORAC) retrieval \citep{Thomas09, Poulsen12} which was previously used
to produce the GRAPE climatology \citep{Thomas09_GRAPE, GRAPE_web}.
Details of the assessment and the results can be found in
\citet{Stengel15}. All three algorithms were then driven with identical MODIS
and AVHRR input data and Era-Interim meteorological background information
from five days during the year 2008 and the results where analyzed with
respect to CLOUDSAT, Calipso and AMSR-E reference data.
This study not only led to a number of recommendations for further retrieval
development but also unveiled that there was no clear winner outperforming the
other algorithms. 
In a nutshell it could be concluded that all algorithms produced acceptable
results, yet challenging situations like multi-layer clouds, especially in
combination with cirrus clouds, created significant difficulties for correct
cloud top and cloud phase detection. 
This problem is generally aggravated by the fact that only the AVHRR-channels from
MODIS are used in the study as well as in the presented work here, in order to
ensure comparability and continuity, and not the sounding channels which could
improve results under the presence of cirrus.
%       ORAC (to 3, but much shorter)
As a result of these findings and due to the general features an optimal
estimation based algorithm offers, which are in close agreement of the
required capabilities laid out above, ORAC was selected to serve as core
retrieval technique for future development and processing of the heritage
dataset.
Right from the onset, development was carried out by several
different groups participating in ESA Cloud CCI, setting the precedent for the
future aspired open community driven improvement.
During this joint development effort, ORAC evolved into the Community OE Cloud
Retrieval for Climate (CC4CL).
As a result, the key components of the retrieval system are
all freely available through a software subversion system, although it has to
be noted that the software is work in progress. 
For a starting point, refer to the
\citet{ORAC_web} website.
